%\documentclass[11pt]{article}          % required in all documents
\documentclass[11pt]{amsart}          % required in all documents
\usepackage{epsfig}                     % a package to include PS figures
\usepackage{amscd}
\usepackage{amssymb}
\usepackage[all]{xy}
\newcommand{\tuborg}{\left\{\begin{array}{ll}}
\newcommand{\sluttuborg}{\end{array}\right.}
\newcommand{\calT}{\mathcal{T}}
\newcommand{\calO}{\mathcal{O}}
\newcommand{\calI}{\mathcal{I}}
\newcommand{\calZ}{\mathcal{Z}}
\newcommand{\calD}{\mathcal{D}}
\newcommand{\calU}{\mathcal{U}}
\newcommand{\calM}{\mathcal{M}}
\newcommand{\calN}{\mathcal{N}}
\newcommand{\calL}{\mathcal{L}}
\newcommand{\calG}{\mathcal{G}}
\newcommand{\calA}{\mathcal{A}}
\newcommand{\calB}{\mathcal{B}}
\newcommand{\calC}{\mathcal{C}}
\newcommand{\calF}{\mathcal{F}}
\newcommand{\calE}{\mathcal{E}}
\newcommand{\calH}{\mathcal{H}}
\newcommand{\calJ}{\mathcal{J}}
\newcommand{\calK}{\mathcal{K}}
\newcommand{\calP}{\mathcal{P}}
\newcommand{\calQ}{\mathcal{Q}}
\newcommand{\cal}{\mathcal}
\newcommand{\bbZ}{\mathbb{Z}}
\newcommand{\bbP}{\mathbb{P}}
\newcommand{\bbQ}{\mathbb{Q}}
\newcommand{\bbR}{\mathbb{R}}
\newcommand{\bbC}{\mathbb{C}}
\newcommand{\bbN}{\mathbb{N}}
\newcommand{\bbA}{\mathbb{A}}
\newcommand{\bbG}{\mathbb{G}}
\newcommand{\bbF}{\mathbb{F}}
\newcommand{\ord}{{\rm ord}}
\newcommand{\coker}{{\rm coker}}
\newcommand{\codim}{{\rm codim}}
\newcommand{\supp}{{\rm Supp}}
\newcommand{\rat}{{\rm Rat}}
\newcommand{\spec}{{\rm Spec}}
\newcommand{\tor}{{\rm Tor}}
\newcommand{\stor}{{\underline{\rm Tor}}}
\newcommand{\p}{\partial}
\newcommand{\proj}{{\rm Proj}}
\newcommand{\fr}{{\rm Frac}}
\newcommand{\ann}{{\rm Ann}}
\newcommand{\ass}{{\rm Ass}}
\newcommand{\pic}{{\rm Pic}}
\newcommand{\ext}{{\rm Ext}}
\newcommand{\shom}{{\underline{\rm Hom}}}
\newcommand{\sext}{{\underline{\rm Ext}}}
\newcommand{\ilim}{{\underset{\longleftarrow}{\lim}}}
\newcommand{\dlim}{{\underset{\longrightarrow}{\lim}}}
\newcommand{\sym}{{\rm Sym}}

\newtheorem{thm}{Theorem}[section]
\newtheorem{lemma}[thm]{Lemma}
\newtheorem{cor}[thm]{Corollary}
\newtheorem{prop}[thm]{Proposition}
\newtheorem{defi}[thm]{Definition} 
\newtheorem{eg}[thm]{Example}
\newtheorem*{claim}{Claim}
\newtheorem*{remark}{Remark}
\newtheorem{remark2}[thm]{Remark}
\newtheorem*{phil}{Philosophy}
\newtheorem{conj}[thm]{Conjecture}
\newtheorem{metaconj}[thm]{Metaconjecture}
\newtheorem{exercise}[thm]{Exercise}
\newtheorem{question}[thm]{Question}
\newtheorem{application}[thm]{Application}

\renewcommand{\theequation}{\arabic{section}.\arabic{thm}.\arabic{equation}}
\renewcommand{\div}{{\rm div}}
\renewcommand{\hom}{{\rm Hom}}

\setlength{\textwidth}{6in}             % Space saving measures
\setlength{\textheight}{9in}
\setlength{\topmargin}{-.5in}
\renewcommand{\baselinestretch}{1}
\setlength{\oddsidemargin}{.25in}
\setlength{\evensidemargin}{.25in}
   
\DeclareSymbolFont{AMSb}{U}{msb}{m}{n}
\DeclareMathSymbol{\N}{\mathbin}{AMSb}{"4E}
\DeclareMathSymbol{\Z}{\mathbin}{AMSb}{"5A}
\DeclareMathSymbol{\R}{\mathbin}{AMSb}{"52}
\DeclareMathSymbol{\Q}{\mathbin}{AMSb}{"51}
\DeclareMathSymbol{\I}{\mathbin}{AMSb}{"49}
\DeclareMathSymbol{\C}{\mathbin}{AMSb}{"43}                                    

\begin{document}

\begin{center}
\bf
\large Robin Hartshorne's Algebraic Geometry Solutions
\end{center}
\begin{center}
by Jinhyun Park
\end{center}
\vskip0.5cm

\section*{Chapter III Cohomology Section 5 The Cohomology of Projective Space}
\subsection*{5.1}
\subsection*{5.2}
\subsection*{5.3}
\subsection*{5.4}
\subsection*{5.5}
\subsection*{5.6}
\begin{enumerate}
\item [(a)] Identify $Q = \bbP^1 \times \bbP^1$ and let $|Y| = \bbP^1 \times *$, $|Z| = * \times \bbP^1$. First, observe that $Q$ is birational to $\bbP^2$ and $h^1 (X, \calO_X)$ is a birational invariant, so, $h^1(Q, \calO_Q) = h^1(\bbP^2, \calO_{\bbP^2}) = 0$. 

\begin{claim}[1] Let $p>0$. Then, $H^1(Q, \calO_Q (p,0)) = 0$.
\end{claim}

\begin{proof} Let $Y = \bbP^1 \times \{ p\mbox{-points} \}$. Then, we have a short exact sequence $0 \to \calO_Q (-p, 0) \to \calO_Q \to \calO_Y \to 0$. Tensor it by $\calO_Q (p,0)$ then, we obtain $0 \to \calO_Q \to \calO_Q (p,0) \to \calO_Y (p,0) \to 0$. Then, from the cohomology long exact sequence, we obtain
$$0 = H^1(Q, \calO_Q) \to H^1(Q, \calO_Q (p,0)) \to \bigoplus_p H^1(\bbP^1, \calO_{\bbP^1} (p |Y|^2)) \to 0,$$ but, $|Y|^2 = 0$, so, by Serre duality, $H^1(\bbP^1, \calO_{\bbP^1}) \simeq H^0 (\bbP^1, \calO_{\bbP^1} (-2) )^* = 0$. Hence, $H^1(Q, \calO_Q(p,0)) = 0$ for $p>0$. This finishes the proof of Claim 1.
\end{proof}

By symmetry, we also have $H^1(Q, \calO_Q (0,q)) = 0$ for $q>0$.

\begin{claim} [2] For all $p \geq 0, q \geq 0$, $H^1(Q, \calO_Q (p,q)) = 0$.
\end{claim}

\begin{proof} If $(p,q) = (0,0)$ or $p=0$ or $q=0$, then, we already know this result, so, assume that $p,q >0$. Tensor the sequence $0 \to \calO_Q (-p, 0) \to \calO_Q  \to \calO_Y \to 0$ with $\calO_Q (p, q)$ to obtain a short exact sequence $0 \to \calO_Q (0,q) \to \calO_Q (p,q) \to \calO_Y (p,q) \to 0$. Then, from the cohomology long exact sequence we have
$$H^1(Q, \calO_Q (0,q)) \to H^1(Q, \calO_Q (p,q)) \to \bigoplus_p H^1(\bbP^1, \calO_{\bbP^1} (p|Y|^2 + q |Y| . |Z|))$$ but $p |Y|^2 + q |Y|.|Z| = q$ and by Serre duality, $H^1(\bbP^1, \calO_{\bbP^1} (q)) \simeq H^0 (\bbP^1, \calO_{\bbP^1} (-q-2))^* = 0$ as $-q-2 <0$. By Claim 1, we know that $H^1 (Q, \calO_Q (0,q) ) = 0$ so, $H^1(Q, \calO_Q (p,q)) = 0$ consequently. This proves the result.
\end{proof}

\begin{claim}[3]For any $p \in \bbZ$, $H^1(Q, \calO_Q (p, -1)) \simeq H^1(Q, \calO_Q (0, -1))$.
\end{claim}


\begin{proof}If $p=0$, it is obvious. First consider the case when $p>0$. From the sequence $0 \to \calO_Q (-p, 0) \to \calO_Q \to \calO_Y \to 0$, by tensoring with $\calO_Q (p, -1)$, we obtain $0 \to \calO_Q (0, -1) \to \calO_Q (p, -1) \to \calO_Y (p, -1) \to 0$. Hence, the long exact sequence gives
$$\bigoplus_p H^0 (\bbP^1, \calO_{\bbP^1} (p|Y|^2 + (-1)|Y|.|Z| = -1) \to H^1(Q, \calO_Q (0, -1))\to H^1(Q, \calO_Q (p, -1)) \to \bigoplus_p H^1( \bbP^1, \calO_{\bbP^1} (-1)).$$ Then, $H^0 (\bbP^1, \calO_{\bbP^1} (-1)) = 0$ and $H^1 (\bbP^1, \calO_{\bbP^1} (-1)) \simeq H^0 (\bbP^1, \calO_{\bbP^1} (-1))^* = 0$, so, $ H^1 (Q, \calO_Q (p, -1)) \simeq H^1(Q, \calO_Q (0, -1))$ indeed.

Now consider the case when $p<0$. let $p' = -p >0$ and let $Y' = \bbP^1 \times \{ p'\mbox{-points} \}$. Then we have $0 \to \calO_Q (-p', 0) \to \calO_Q \to \calO_{Y'} \to 0$ and by tensoring with $\calO_Q (0, -1)$, we obtain $0 \to \calO_Q (-p', -1) \to \calO_Q (0, -1) \to \calO_{Y'} (0, -1) \to 0$. Hence, the long exact sequence gives us
$$\bigoplus_{p'} H^0 (\bbP^1, \calO_{\bbP^1} (-1)) \to H^1(Q, \calO_Q (-p', -1)) \to H^1(Q, \calO_Q (0, -1)) \to \bigoplus_{p'} H^1(\bbP^1, \calO_{\bbP^1} (-1))$$ and $H^0 (\bbP^1, \calO_{\bbP^1} (-1)) = H^1 (\bbP^1, \calO_{\bbP^1} (-1)) = 0$. This shows that $H^1 (Q, \calO_Q (p, -1)) \simeq H^1(Q, \calP_Q (0, -1))$ for $p<0$.
\end{proof}

\begin{claim}[4] \begin{enumerate}
\item [(i)] $H^1(Q, \calO_Q (0, q)) \not = 0$ if $q \leq -2$.
\item [(ii)] $H^1 (Q, \calO_Q (0, -1)) = 0$.
\end{enumerate}
\end{claim}

\begin{proof} Let $p>0$. From $0 \to \calO_Q (-p, 0) \to \calO_Q \to \calO_Y \to 0$, by tensoring with $\calO_Q (0, q)$, we obtain $0 \to \calO_Q (-p, q) \to \calO_Q (0,q) \to \calO_Y (0,q) \to 0$ so that the long exact sequence gives
$$\bigoplus_p H^0 (\bbP^1, \calO_{\bbP^1} (q)) \to H^1(Q, \calO_Q (-p, q) )\to H^1(Q, \calO_Q (0, q)) \to \bigoplus_p H^1 (\bbP^1, \calO_{\bbP^1} (q)) \to 0.$$

When $q \leq -2$, $H^0 (\bbP^1, \calO_{\bbP^1} (-1) ) = 0$ and $h^1 (\bbP^1 , \calO_{\bbP^1} (q)) = h^0 (\bbP^1, \calO_{\bbP^1} (-q-2)) >0$ so that $H^1 (0, q) \not = 0$. This proves (i) and by symmetry we also have $H^1 (p, 0) \not = 0$ if $p \leq -2$. This proves (3).

When $p = 1, q=0$, we have
$$k = H^0 (Q, \calO_Q) \overset{\simeq}{\to} H^0 (\bbP^1, \calO_{\bbP^1}) = k \to H^1 (Q, \calO_Q (-1, 0)) \to H^1 (Q, \calO_Q) = 0$$ so that $H^1(Q, \calO_Q (-1, 0)) = 0$. This proves (ii) and similarly we have $H^1 (Q, \calO_Q (0, -1)) = 0$.
\end{proof}

Now, we prove (2). From V. 1.4.4, the canonical line bundle $K \simeq \calO_Q (-2, -2)$, so, when $a,b<0$, by Serre duality, 
$$H^1(Q, \calO_Q (a,b)) \simeq H^1(Q, \calO_Q (-a-2, -b-2))^*.$$

If $a, b \leq -2$, then, by Claim 2, this group vanishes.

In case $(a,b) = (0, -1), (-1, 0), (-2, -1), (-1,-2), (-1,-1)$, the previous claims already show it. Hence, it is $0$ for any $a,b<0$. This proves (2). (1) is trivial once we have (2) and the previous claims. 

\item [(b)]
\begin{enumerate}
	\item [(1)] For $Y$, $0 \to \calO_Q (-Y) \to \calO_Q \to \calO_Y\to 0$ is exact and $\calO_Q (-Y) \simeq \calO_Q (-a, -b)$. Thus, 
	$$0 \to H^0 (Q, \calP_Q (-a, -b)) \to H^0 (Q, \calO_Q) \to H^0 (Y, \calO_Y) \to H^1(Q, \calO_Q (-a, -b)) \to 0$$ so, $H^0 (Y, \calO_Y) \simeq H^0 (Q, \calO_Q) \simeq k$. Hence $Y$ has only $1$ connected component, i.e. connected.
	
	\item [(2)] Let $\calL$ be a line bundle on $Q$ of type $(a,b)$ with $a>0, b>0$. Then, by II. 7.6.2, $\calL$ is very ample so that it gives an embedding of $Q$ into a projective space $\bbP^N$. Then, by Bertini's theorem (II, 8.18), there is a hyperplane $H \subset \bbP^N$ whose intersection with $Q$ is a nonsingular projective curve $Y$ and this $\calO_Q (Y)$ is isomorphic to $\calL$, i.e. $Y$ is of type $(a,b)$.
	
	\item [(3)] By Ex. II 5.14-(d), $X \subset \bbP_A ^r$ is projectively normal if and only if it is normal and for all $n \geq 0$, the natural map $\Gamma(\bbP^r, \calO_{\bbP^r} (n)) \to \Gamma(X, \calO_X (n))$ is surjective. We will use this.
	
	Since we have a sequence of closed embeddings $Y \hookrightarrow Q \hookrightarrow \bbP^3$, it gives a commutative diagram
	$$\xymatrix{ \Gamma( \bbP^3, \calO_{\bbP^3} (n)) \ar[dr] \ar[rr] & & \Gamma(Y, \calO_Y (n)) \\ & \Gamma (Q, \calO_Q (n)) \ar[ru] & }$$ so, if, $\Gamma (Q, \calO_Q(n)) \to \Gamma(Y, \calO_Y (n))$ is not surjective, then, $\Gamma(\bbP^3, \calO_{\bbP^3}(n)) \to \Gamma(Y, \calO_Y (n))$ cannot be surjective.
	
	On the other hand, since $Q = V(xy- zw) \subset \bbP^3$, the ideal sheaf of Q $\calI_Q \simeq \calO_{\bbP^3} (-2)$ so that the sequence $0 \to \calO_{\bbP^3} (-2) \to \calO_{\bbP^3} \to \calO_Q \to 0$ is exact. Hence, by tensoring with $\calO_{\bbP^3} (n)$, we have $0 \to \calO_{\bbP^3} (n-2) \to \calO_{\bbP^3} (n) \to \calP_Q (n) \to 0$ whose cohomology long exact sequence gives
	$$H^0 (\bbP^3, \calO_{\bbP^3} (n)) \to H^0 (Q, \calO_Q (n)) \to H^1 (\bbP^3, \calO_{\bbP^3} (n-2)) = 0.$$ Consequently, the map $\Gamma (\bbP^3, \calO_{\bbP^3} (n)) \to \Gamma (Q, \calO_Q (n))$ is always surjective and it implies that $\Gamma (\bbP^3, \calO_{\bbP^3}(n)) \to \Gamma(Y, \calO_Y (n))$ is surjective if and only if $\Gamma (Q, \calO_Q (n)) \to \Gamma (Y, \calO_Y (n)))$ is surjective if and only if $Y \subset \bbP^3$ is projectively normal, because being nonsingular, $Y$ is already normal.
	
	Hence, it remains to show that $\Gamma (Q, \calO_Q (n)) \to \Gamma (Y, \calO_Y (n))$ is surjective if and only if $|a-b|\leq 1$.
	
	($\Leftarrow$) Suppose that $|a-b|\leq 1$. Then, from $0 \to \calO_Q (-a, -b) \to \calO_Q \to \calO_Y \to 0$, we obtain $0 \to \calO_Q (n-a, n-b) \to \calO_Q (n,n) \to \calO_Y (n) \to 0$ which gives us
	$$\Gamma (Q, \calO_Q (n)) \to \Gamma (Y, \calO_Y (n)) \to H^1(Q, \calO_Q (n-a, n-b).$$ But, $|a-b | \leq 1$ means $|(n-a) - (n-b) |\leq 1$ so, by part (a) - (1), $H^1 (Q, \calO_Q (n-a, n-b))$ vanishes and the natural map is surjective.
	
	($\Rightarrow$) Conversely, suppose that the natural map is surjective for all $n \geq 0$. Then, the same sequence gives
	$$\Gamma (Q, \calO_Q(n)) \to \Gamma(Y, \calO_Y (n)) \to H^1(Q, \calO_Q (n-a, n-b)) \to H^1(Q, \calO_Q (n,n))$$ where the last one is $0$ by Claim 2 of (a) and the first map is surjective. Hence, we must have $H^1(Q, \calO_Q (n-a, n-b) ) = 0$ for all $n \geq 0$.
	
	Toward contradiction, so, suppose that $|a-b | \geq 2$, i.e. $a \geq b+2$ or $b \geq a + 2$. For the first case, when $n = b$, $n-a \leq -2$ so that by (a)- (3), we have $H^1 (Q, \calO_Q (n-a, n-b)) \not = 0$, which is a contradiction. For the second case, we will have the same contradiction. Hence $|a-b | \leq 1$.
	
	Hence, a nonsingular $Y \subset Q$ of type $(a,b)$ with $a,b >0$ is projectively normal in $\bbP^3$ if and only if $|a-b | \leq 1$.
	\end{enumerate}
\item [(c)] First, we reduce this problem to a nonsingular $Y$. By part (b)-(2), $Y$ is linearly (hence rationally) equivalent to a nonsingular projective curve lying on $Q$ and this new curve has the same bidegree. Also, since this is a rational equivalence, they belong to the same flat family, so, the arithmetic genera are unchanged (which are defined to be $h^1(Y, \calO_Y)$). Hence, we may replace $Y$ by its linearly equivalent nonsingular $Y$. Then, for this $Y$, the arithmetic genus $p_a (Y) = p_g (Y)$, the geometric genus, and we can compute it in terms of $a,b$ as follows: $\calO_Q (Y) = \calO_Q (a,b)$ and the first Chern class $c_1 (N_{Q/Y}) = \deg _Y (N_{Q/Y}) = Y . Y = (ah + bk)^2 = ab (h.k) + ba (k.h) = 2ab$ where $h, k$ are generators of $\pic Q \simeq \bbZ \oplus \bbZ$ with intersection product $h^2 = k^2 = 0$, $h.k = k.h = 1$. 

On the other hand, $T_{\bbP^1 \times \bbP^1 } = \left( \Omega _{\bbP^1 \times \bbP^1} ^1 \right)^*$ implies that $c_1 \left( T_{\bbP^1 \times \bbP^1 } \right) = c_1 \left( \wedge ^2 T_{\bbP^1 \times \bbP^1 } \right) = c_1 \left( K_{\bbP^1 \times \bbP^1 } ^* \right) = - (c_1 (K_{\bbP^1}), c_1 (K_{\bbP^1})) = - (2 \cdot 0 - 2, 2 \cdot 0 - 2 ) = (2,2) = 2h + 2k$ and so, $c_1 (T_Q |_Y) = \deg _Y \left( T_Q \otimes_{\calO_Q} \calO_C \right) = \deg_Y \left( \wedge^2 T_Q \otimes_{\calO_Q} \calO_C \right) = [K_Q ^*] \cdot (ah + bk) = (2h + 2k) \cdot ( ah + bk) = 2 (a+b)$.

Of course, $c_1 (T_Y) = - \deg _Y (K_Y) = - (2g - 2) = 2 - 2g$. Hence the short exact sequence
$$0 \to T_Y \to T_Q |_Y \to N_{Q/Y} \to 0$$ gives $c_1 (T_Y) + c_1 (N_{Q/Y}) = c_1 (T_Q |Y)$ and it is equivalent to $2 (a+b) = 2 - 2g + 2ab$, i.e. $g = ab - a - b +1 = (a-1)(b-1)$. This proves the result.
\end{enumerate}

\subsection*{5.7}
\subsection*{5.8}
\subsection*{5.9}
\subsection*{5.10}

\end{document}

