%%%%%%%%%%%%%%%%%%%%%%%%%%
%% Homework assignment 2
%%%%

\documentclass[12pt]{article}
\textwidth=1.2\textwidth
\textheight=1.3\textheight
\hoffset=-.7in
\voffset=-1.28in
\usepackage{amsmath}
\usepackage{amsthm}
\usepackage{amsopn}
\usepackage{amscd}

\font\bbb=msbm10 scaled \magstep 1
\font\german=eufm10 scaled \magstep 1
\font\script=rsfs10 scaled \magstep 1

\newcommand{\nd}{\not\!\!|}
\newcommand{\m}{\mathbf{m}}
\newcommand{\cross}{\times}
\newcommand{\diff}{\Omega_{B/A}}
\newcommand{\injects}{\hookrightarrow}
\newcommand{\pt}{\mbox{\rm pt}}
\newcommand{\dual}{\vee}
\newcommand{\bF}{\mathbf{F}}
\newcommand{\bZ}{\mathbf{Z}}
\newcommand{\bR}{\mathbf{R}}
\newcommand{\bQ}{\mathbf{Q}}
\newcommand{\bC}{\mathbf{C}}
\newcommand{\bA}{\mathbf{A}}
\newcommand{\bP}{\mathbf{P}}
\newcommand{\sP}{\mathcal{P}}
\newcommand{\M}{\mathcal{M}}
\newcommand{\sO}{\mathcal{O}}
\newcommand{\sox}{\sO_X}
\newcommand{\soc}{\sO_C}
\newcommand{\soy}{\sO_Y}
\newcommand{\ox}{\omega_X}
\newcommand{\so}{\sO}
%\newcommand{\sF}{\mbox{\script F}}
\newcommand{\sQ}{\mathcal{Q}}
\newcommand{\sR}{\mathcal{R}}
\newcommand{\sE}{\mathcal{E}}
\newcommand{\E}{\mathcal{E}}
\newcommand{\sF}{\mathcal{F}}
%\newcommand{\sL}{\mbox{\script L}}
\newcommand{\sL}{\mathcal{L}}
\newcommand{\sM}{\mathcal{M}}
\newcommand{\sN}{\mathcal{N}}
\newcommand{\sA}{\mathcal{A}}
\newcommand{\sD}{\mathcal{D}}
\newcommand{\sC}{\mathcal{C}}
\newcommand{\sG}{\mathcal{G}}
\newcommand{\sB}{\mathcal{B}}
\newcommand{\sK}{\mathcal{K}}
\newcommand{\sH}{\mathcal{H}}
\newcommand{\sI}{\mathcal{I}}
\newcommand{\sU}{\mbox{\german U}}
\newcommand{\gm}{\mbox{\german m}}
\newcommand{\isom}{\cong}
\newcommand{\tensor}{\otimes}
\newcommand{\into}{\rightarrow}
\newcommand{\soq}{\sO_{Q}}
\newcommand{\soxp}{\sO_{X,p}}
\newcommand{\Cone}{Q_{\text{cone}}}
\newcommand{\cech}{\v{C}ech}
\newcommand{\cH}{\text{\v{H}}}
\newcommand{\intersect}{\cap}
\newcommand{\union}{\cup}
\newcommand{\iso}{\xrightarrow{\sim}}
\newcommand{\qone}{Q_{\text{one}}}
\newcommand{\qns}{Q_{\text{ns}}}
\newcommand{\OX}{\Omega_X}

\newcommand{\F}{\sF}
\renewcommand{\P}{\bP}
\newcommand{\A}{\bA}
\newcommand{\C}{\bC}
\newcommand{\Q}{\bQ}
\newcommand{\R}{\bR}
\newcommand{\Z}{\bZ}

\newcommand{\chose}[2]{ {{#1} \choose {#2}} }
\renewcommand{\L}{\mathcal{L}}


\DeclareMathOperator{\cd}{cd}
\DeclareMathOperator{\Ob}{Ob}
\DeclareMathOperator{\Char}{char}
\DeclareMathOperator{\aut}{Aut}
\DeclareMathOperator{\End}{End}
\DeclareMathOperator{\gl}{GL}
\DeclareMathOperator{\slm}{SL}
\DeclareMathOperator{\supp}{supp}
\DeclareMathOperator{\spec}{Spec}
\DeclareMathOperator{\Spec}{Spec}
\DeclareMathOperator{\ext}{Ext}
\DeclareMathOperator{\Ext}{Ext}
\DeclareMathOperator{\tor}{Tor}
\DeclareMathOperator{\Hom}{Hom}
\DeclareMathOperator{\Aut}{Aut}
\DeclareMathOperator{\PGL}{PGL}
\DeclareMathOperator{\shom}{\mathcal{H}om}
\DeclareMathOperator{\sHom}{\mathcal{H}om}
\DeclareMathOperator{\sext}{\mathcal{E}xt}
\DeclareMathOperator{\proj}{Proj}
\DeclareMathOperator{\Pic}{Pic}
\DeclareMathOperator{\pic}{Pic}
\DeclareMathOperator{\pico}{Pic^0}
\DeclareMathOperator{\gal}{Gal}
\DeclareMathOperator{\imag}{Im}  
\DeclareMathOperator{\Id}{Id}  
\DeclareMathOperator{\Ab}{\mathbf{Ab}}
\DeclareMathOperator{\Mod}{\mathbf{Mod}}
\DeclareMathOperator{\Coh}{\mathbf{Coh}}
\DeclareMathOperator{\Qco}{\mathbf{Qco}}
\DeclareMathOperator{\hd}{hd}
\DeclareMathOperator{\depth}{depth}
\DeclareMathOperator{\trdeg}{trdeg}
\DeclareMathOperator{\rank}{rank}
\DeclareMathOperator{\Tr}{Tr}
\DeclareMathOperator{\length}{length}
\DeclareMathOperator{\Hilb}{Hilb}
\DeclareMathOperator{\Sch}{\mbox{\bfseries Sch}}
\DeclareMathOperator{\Set}{\mbox{\bfseries Set}}
\DeclareMathOperator{\Grp}{\mbox{\bfseries Grp}}
\DeclareMathOperator{\id}{id}
\DeclareMathOperator{\codim}{codim}
\DeclareMathOperator{\Var}{Var}

\theoremstyle{plain}
\newtheorem{thm}{Theorem}[section]
\newtheorem{prop}[thm]{Proposition}
\newtheorem{claim}[thm]{Claim}
\newtheorem{cor}[thm]{Corollary}
\newtheorem{fact}[thm]{Fact}
\newtheorem{lem}[thm]{Lemma}
\newtheorem{ques}[thm]{Question}
\newtheorem{conj}[thm]{Conjecture}

\theoremstyle{definition} 
\newtheorem{defn}[thm]{Definition}

\theoremstyle{remark}
\newtheorem{remark}[thm]{Remark}
\newtheorem{exercise}[thm]{Exercise}
\newtheorem{example}[thm]{Example}
\author{William A. Stein}
\title{Homework 2, MAT256B\\II.8.4, III.6.8, III.7.1, III.7.3}
\date{April 2, 1996}
\begin{document}
\maketitle

\section{Exercise II.8.4}
{\em Complete Intersections in $\P^n$. 
A closed subscheme $Y$ of $\P^n_k$ is called a {\em (strict, global)
complete intersection} if the homogenous ideal $I$ of $Y$ in
$S=k[x_0,\ldots,x_n]$ can be generated by $r$ elements where
$r=\codim(Y,\P^n)$.} 

{\em (a) Let $Y$ be a closed subscheme of codimension $r$ in $\P^n$. Then
$Y$ is a complete intersection iff there are hypersurfaces (i.e., locally
principal subschemes of codimension $1$) $H_1,\ldots,H_r$, such that
$Y=H_1\intersect\cdots\intersect H_r$ {\em as schemes}, i.e.,
$\sI_Y=\sI_{H_1}+\cdots+\sI_{H_r}$.}

($\Rightarrow$) By (II, Ex 5.14) $I$ is defined to be $\Gamma_{*}(\sI_Y)$.
By (II, 5.15), $\tilde{I}\isom\sI_Y$. Write $I=(f_1,\ldots,f_r)$, then
since localization commutes with taking sums,
$$\sI_Y=(f_1,\ldots,f_r)\tilde{ }=((f_1)+\cdots+(f_r))\tilde{ }=(f_1)\tilde{ }+\cdots+(f_r)\tilde{ }.$$
Let $H_i$ be the locally principal closed subscheme of codimension
$1$ determined by the ideal sheaf $(f_i)\tilde{ }$.  Then $Y$ is the
intersection of the $H_i$. 

($\Leftarrow$) Someone suggested I should apply unmixedness and 
primary decomposition to some ideal somewhere and use the fact 
that a saturated ideal doesn't have primary components corresponding 
to the irrelevant ideal or something like that. NOT DONE.  

{\em (b) If $Y$ is a complete intersection of dimension $\geq 1$ in
$\P^n$, and if $Y$ is normal, then $Y$ is projectively normal (Ex. 5.14).}

Let $Z$ be the cone over $Y$, then $A(Z)=S/I(Y)$. By
(I, Ex. 3.17d), $A(Z)$ is integrally closed iff $Z$ is
normal. By definition $A(Z)$ is integrally closed iff
$Y$ is projectively normal. Thus we must show that $Z$
is normal. Since $Y$ is a complete intersection, $I(Y)=(f_1,\ldots,f_r)$
so $Z$ is a complete intersection subscheme of $\bA^{n+1}$. By
(II, 8.23) $Z$ is normal iff $Z$ is regular in codimension $1$. 
Also by (II, 8.23) $Y$ is regular in codimension $1$ because we
have assumed $Y$ is normal. But $Y$ regular in codimension $1$ implies
$Z$ regular in codimension $1$. [We used this last semester in 
(II, Ex. 6.3d). Intuitively, the only singularity in $Z$ not in $Y$
is the cone point which has codimension $>1$. This is because $Z$
is locally $U_i\cross\bA^1$.] 

{\em (c) With the same hypothesis as in (b), conclude that for all
$\ell\geq 0$, the natural map $\Gamma(\P^n,\sO_{\P^n}(\ell))\into\Gamma(Y,\sO_Y(\ell))$
is surjective. In particular, taking $\ell=0$, show that $Y$ is connected.}

That the map $\Gamma(\P^n,\sO_{\P^n}(\ell))\into\Gamma(Y,\sO_Y(\ell))$
is surjective is just the statement of (II, Ex. 5.14d). When $\ell=0$
this says that 
$k=\Gamma(\P^n,\sO_{\P^n}(\ell))$ surjects onto $\Gamma(Y,\sO_Y(\ell))$
so $\dim\Gamma(Y,\sO_Y)\leq 1$ and hence $Y$ is connected. [If $Y$ were
not connected then $\Gamma(Y,\sO_Y)=k\oplus\cdots\oplus k$ where the
number of direct summands equals the number of components of $Y$.]

{\em (d) Now suppose given integers $d_1,\ldots,d_r\geq 1$, with
$r<n$. Use Bertini's theorem (8.18) to show that there exists nonsingular
hypersurfaces $H_1,\ldots,H_r$ in $\P^n$, with $\deg H_i=d_i$, such that
the scheme $Y=H_1\intersect\cdots\intersect H_r$ is irreducible
and nonsingular in codimension $r$ in $\P^n$.}

[To apply Bertini's theorem we must assume $k$ is algebraically closed.
I'm going to make this assumption now. Maybe there is a way around this?] 

Let $\P_k^n\hookrightarrow\P_k^{d_1\text{-uple}}$ 
be the $d_1$-uple embedding of $\P_k^n$.  Use Bertini's theorem to choose
a hyperplane in
$\P_k^{d_1\text{-uple}}$ which has nonsingular intersection with
the image of $\P_k^n$. It pulls back to a degree $d_1$ nonsingular
hypersurface $H_1$ in $\P_k^n$. If $r>1$ consider the $d_2$-uple embedding
$\P_k^n\hookrightarrow\P_k^{d_2\text{-uple}}$. 
The image of $H_1$ is a nonsingular variety in $\P_k^{d_2{\text-uple}}$
of dimension $\geq 2$. By Bertinni's theorem there is a hyperplane
in $\P_k^{d_2-uple}$ whose intersection with the image of $H_1$ is nonsingular
and of dimension one less than $H_1$. Pulling back we obtain a hypersurface
$H_2$ such that $H_1\intersect H_2$ is nonsingular and $H_2$ has degree
$d_2$. Continuing inductively in this way and noting that
$\dim H_1\intersect\cdots\intersect H_{r-1}\geq 2$ (since 
$r<n$) completes the proof. 

{\em (e) If $Y$ is a nonsingular complete intersection as in (d) 
show that $\omega_Y\isom\sO_Y(\sum d_i-n-1)$.}

By (III, 8.20) $\omega_{H_1}\isom\omega_{\P^n}\tensor\sL(H_1)\tensor\sO_{H_1}$.
By the explicit computation of $Cl\P^n$ (II, 6.17) we know that
$\sL(H_1)\isom\sO_{\P^n}(d_1)$. Thus 
$$\omega_{H_1}\isom\sO_{\P^n}(-n-1)\tensor\sO_{\P^n}(d_1)\tensor\sO_{H_1}\isom\sO_{H_1}(d_1-n-1).$$ 

By (8.20) we have that
$$\omega_{H_1\intersect H_2}\isom\omega_{H_1}\tensor\sL(H_2.H_1)\tensor\sO_{H_1\intersect H_2}.$$
We know that $H_2\sim d_2\P^{n-1}$ (linear equivalence) so
by (II, 6.2b) this implies $H_2.H_1\sim d_2\P^{n-1}.H_1$. 
But $d_2\P^{n-1}.H_1$ corresponds to the invertible sheaf (see (II, Ex 6.8c)
$\sO_{H_1}(d_2)$. Thus
$$\omega_{H_1\intersect H_2}\isom\sO_{H_1}(d_1-n-1)\tensor\sO_{H_1}(d_2)\tensor\sO_{H_1\intersect H_2}\isom\sO_{H_1\intersect H_2}(d_1+d_2-n-1).$$
Repeating this argument inductively yields the desired isomorphism. 

{\em (f) If $Y$ is a nonsingular hypersurface of degree $d$ in $\P^n$,
use (c) and (e) above to show that $p_g(Y)=\chose{d-1}{n}$. Thus
$p_g(Y)=p_a(Y)$ (I, Ex. 7.2).} 

By definition $p_g(Y)=\dim_k\Gamma(Y,\omega_Y)$. By (e),
$\omega_Y\isom\sO(d-n-1)$ and by (c) the natural map
$$\Gamma(X,\sox(d-n-1))\into\Gamma(Y,\soy(d-n-1))$$
is surjective. We show that it is also injective.
By (III, 5.5a) an element $f\in\Gamma(X,\sox(d-n-1))$ can be represented as
a homogeneous polynomial of degree $d-n-1$.
Now $f$ maps to $0$ in $\Gamma(Y,\soy(d-n-1))$ iff
$f$ vanishes on $Y$, that is to say, $Y\subset Z(f)$. But
$\deg Y=d>d-n-1=\deg f$ so $Y$ can not be contained in the hypersurface
$Z(f)$ unless $f=0$. [Proof: $Y=Z(g)\subset Z(f)$ implies
$(f)\subset (g)$ so $f$ is a multiple of $g$, but $g$ has degree
strictly greater than $f$ so must be $0$.]
Thus 
$$\dim_k \Gamma(Y,\soy(d-n-1))=\dim_k \Gamma(X,\sox(d-n-1))
             =\chose{d-1}{n}$$
since the number of monomoials in $k[x_0,\ldots, x_n]$ of degree
$d-n-1$ is $\chose{d-1}{n}$
as desired. 

{\em (g) If $Y$ is a nonsingular curve in $\P^3$, which is a complete
intersection of nonsingular surfaces of degrees $d$, $e$, then
$p_g(Y)=\frac{1}{2}de(d+e-4)+1$. Again the geometric genus is
the same as the arithmetic genus (I, Ex. 7.2).}

Let $H$ be the hypersurface of degree $d$.
There is an exact sequence
$$0\into\sO_{\P^n}(-d)\into\sO_{\P^n}\into\sO_H\into 0.$$
Twisting by $a$ and computing dimensions we see that
$$\dim\sO_H(a)=\dim\sO_{\P^n}(a)-\dim\sO_{\P^n}(a-d)=\chose{3+a}{3}-\chose{3+a-d}{3}.$$

Using reasoning like that in (e) we obtain an exact sequence
$$0\into\sO_H(-e)\into\sO_H\into\sO_Y\into 0.$$
Twisting by $e+d-4$ yields the exact sequence
$$0\into\sO_H(d-4)\into\sO_H(e+d-4)\into\sO_Y(e+d-4)\into 0.$$
Applying the above explicit computation of $\dim\sO_H(a)$ we see that
$$\dim\soy(e+d-4)=\chose{e+d-1}{3}-\chose{e-1}{3}-\chose{d-1}{3}+chose{-1}{3}.$$
After some algebra the latter expression becomes
$\frac{1}{2}ed(e+d-4)+1$, as desired. 

[Comment 1: We could have also solved (f) using this method.]

[Comment 2: Serre duality gives another solution. By (III, 7.12.4)
$p_g(Y)=\dim H^0(Y,\omega_Y)=\dim H^1(Y,\soy)=p_a(Y).$
But by (I, Ex. 7.2d) $p_a(Y)=\frac{1}{2}de(d+e-4)+1$.]


\section{Exercise III.6.8}
{\em Prove the following theorem of
Kleiman: if $X$ is a noetherian, integral, seperated,
locally factorial scheme, then every coherent sheaf on 
$X$ is a quotient of a locally free sheaf (of finite rank).} 

{\em (a) First show that open sets of the form $X_s$, for
various $s\in\Gamma(X,\sL)$ and various invertible sheaves
$\sL$ on $X$, form a base for the topology of $X$.} 

Let $x\in U\subset X$ with $U$ open. 

{\em Case 1. $W=X-U$ is irreducible.}
Since $x\not\in W$, $\so_x\not\subset \sO_W$. 
[This assertion is a matter of some difficulty among the
others working on this problem. It is not hard to see when 
$X$ is a variety in the classical sense. But in the more general
situation it isn't at all clear and may use the hypothesis that 
$X$ is seperated in an essential way. For example, the affine line
with a doubled origin has two different local rings which are equal.
I'm not sure how to resolve this but there was some talk of using
the valuative criterion for seperatedness. PUT CORRECT SOLUTION HERE 
AFTERWARDS.]
Thus let $h\in K$ be a rational
function such that $h\not\in\sO_W$ but $h\in\so_x$. Let $(h)=D_1-D_2$
with $D_1$=zeros of $h$ and $D_2$=poles of $h$. Since $h\in\so_x$, we
have $x\not\in\Var(D_2)$=the underlying scheme of the effective
divisor $D_2$. (This is because $h$ can't have a pole at $x$.) Furthermore
$y\in W$ implies $\so_y\subset\so_W$ so $h\not\in\sO_y$ thus $y\in\Var(D_2)$ 
(this is because $X$
is factorial so $\sO_y$ is integrally closed so $v_y(h)<0$ iff
$h\not\in\sO_y$.) Thus $W\subset\Var(D_2)$. Since $X$ is factorial and
$D_2$ is effective (II, 6.11) implies
$D_2$ corresponds to an effective Cartier divisor and hence there exists
an open cover $\sU=(U_i)$ of $X$ and rational functions $h_i\in K$
such that $h_i|U_i\in\sO_{U_i}$ and $(h_i)=D_2$ on $U_i$. Since
$\frac{h_i}{h_j}\in\so_x(U_i\intersect U_j)^{*}$ and $X$ is normal,
$(\frac{h_i}{h_j})=0$. Let $\sL$ be the locally free invertible sheaf
represented by the Cartier divisor $(U_i,h_i)$ (so $\L$ is locally
generated by $1/h_i$ on $U_i$), and let $u_i:\sL|U_i\into\so_x|U_i$
be the isomorphism given by multiplication by $h_i$. Define
$s(y)=u_i^{-1}(h_i(y))$ for $y\in U_i$. By this we mean $u_i^{-1}$ of
the map $y\mapsto h_i(y)$, i.e., $s$ is the glueing of the inverse images
of the $h_i\in\so_x(u_i)$. Thus $s$ is a section of $\L$ such that
$X_s\intersect U_i=U_i-\Var(D_2)$. Thus
$X_s=X-\Var(D_2)\subset U$. 

{\em Case 2. $W=X-U$ is reducible.}
Using the fact that $X$ is noetherian write
$W=Z_1\union \cdots \union Z_n$. From case 1 we know that
there exists invertible sheaves $\L_1,\ldots,\L_n$ and sections
$s_i\in\Gamma(X,\L_i)$, $i=1,\ldots,n$ such that $x\in X_{s_i}\subset X-Z_i$. 
Let $s=s_1\tensor\cdots\tensor s_n\in\Gamma(X,\L_1\tensor\cdots\tensor\L_n)$. 
Then $X_s=\intersect_{i=1}^n X_{s_i}$ hence $x\in X_s\subset U$. 

[[This proof was copied from Borelli's paper with little modification. One
danger is that the corresponding theorem in Borelli's paper assumes $X$ to be
a factorial {\em variety}, not a more general scheme as above. Part (b)
below was not in Borelli.]]

{\em (b) Now use (II, 5.14) to show that any coherent sheaf is a quotient
of a direct sum $\oplus\sL_i^{n_i}$ for various invertible sheaves
$\sL_i$ and various integers $n_i$. }

Let $\sF$ be a coherent sheaf on $X$. 
Let $U$ be an open set on which $\sF_{|U}\isom\tilde{M}$. 
Suppose $X_f\subset U$ where $f$ is a global section of some
invertible sheaf $\sL$. Our strategy is to construct an appropriate
map $\oplus\sL_i^{n_i}\into \sF$ which is surjective when 
restricted to $X_f$, then use the fact that $X$ is noetherian and
that the $X_f$ form a basis for the topology on $X$ to cover $X$
which such $X_f$ and then take the sum of all the resulting maps.

Let $m_1,\ldots,m_r$ generate $M$. Let $t_1,\ldots,t_r$ be the
restrictions of the $m_i$ to $X_f$. By (II, 5.14b) there exists $n$
so that 
$$t_1 f^n,\ldots,t_r f^n\in\Gamma(X_{f^n},\sF\tensor\sL^{\tensor n})$$
extend to global sections $s_1,\ldots,s_n$ of 
$\Gamma(X,\sF\tensor\sL^{\tensor n})$. Define a map
$$\oplus_{i=1}^n \sox\into\sF\tensor\sL^{\tensor n}$$
by sending $(0,\ldots,0,1,0,\ldots,0)$ (1 in the $i$th position only)
to $s_i$. Then tensoring with
$(\sL^{\tensor n})^{-1} =(\sL^{-1})^{\tensor n}$
we obtain a map
$$\Theta:\oplus_{i=1}^n (\sL^{-1})^{\tensor n}\into\sF.$$
The map $\Theta$ is surjective when restricted to $X_f$. 
To see this let $p$ be a point of $X_f$. The stalk of
$\sF$ at $p$ is generated by the stalks of $m_1,\ldots,m_n$ 
at $p$. Since the $t_i$ are all in the image of the map $\Theta$ and
the stalks of the $t_i$ at $p$ are the same as the stalks of the $m_i$
at $p$ it follows that the stalks of the $m_i$ are all in the image 
under $\Theta$ of the stalk of $\oplus_{i=1}^n(\sL^{-1})^{\tensor n}$
at $p$. 

Take the direct sum of all such maps over a suitable open cover
$(U)$ of $X$ and suitable open covers $(X_f)$ of each $U$. Since
$X$ is noetherian we can arrange it so this sum is finite. \

\section{Exercise III.7.1}
{\em Let $X$ be an integral projective
scheme of dimension $\geq 1$ over a field $k$, and let 
$\sL$ be an ample invertible sheaf $X$. Then $H^0(X,\sL^{-1})=0$.}

\begin{lem} If $\M\neq\sox$ is an invertible sheaf which is generated by
its global sections then $H^0(X,\M^{-1})=0$. \end{lem}
\begin{proof}
By the proof of (II 6.12) $\M^{-1}=\M^{\dual}=\shom(\M,\sox)$. 
Thus we must show that $\Gamma(X,\shom(\M,\sox))=0$, i.e., that
$\hom_{\sox}(\M,\sox)=0$. Since $\M$ is generated by global
sections $(m_i)$ to give a morphism $f:\M\into\sox$ is the same
as to give the images $\alpha_i=f(m_i)\in\Gamma(X,\sox)$ of the
$m_i$. Since $X$ is integral and projective $\Gamma(X,\sox)=k$
so the $\alpha_i$ all lie in $k$. Thus if $f$
is nonzero then some $\alpha_i\neq 0$ so $\frac{1}{\alpha_i}m_i\mapsto 1$.
Let $t=\frac{1}{\alpha_i}m_i\in\Gamma(X,\M)$. Let $p$ be any
point of $X$. The map $f_p:\M_p\into\soxp$ sends $t_p$ to
$1$ so it is surjective being a map of free $\soxp$-modules (and
since $\soxp$ is generated by $1$ as an $\soxp$-module). On the
other hand $\M_p$ is free of rank $1$ over the integral domain
$\soxp$ so $f$ must be injective. Indeed, if $\M_p\isom\soxp\cdot g$
for some $g$ and $ag\mapsto 0$ then $af(g)=0$ so since $\soxp$ is a domain,
$a=0$ or $f(g)=0$. But $f(g)\neq 0$ since $f$ is surjective so $a=0$ 
and so $ag=0$ whence $f$ is injective. Therefore $f$ is an isomorphism
since it is an isomorphism on stalks. Thus $\M\isom\sox$ contrary
to our assumption that $\M\not\isom\sox$ so there
can be no nonzero $f$ in $\hom_{\sox}(\M,\sox)=\Gamma(X,\M^{-1})$,
as desired.
\end{proof}

Suppose that $\L$ is ample. If $\L=\sox$ then $\L$ can not be ample,
for if $\sox$ is ample then since $\sox^{\tensor n}=\sox$ for any $n\geq 1$
it follows by (II.7.5) that $\sox$ is very ample. This means that there
is an immersion $i:X\hookrightarrow\P_k^n$ where $n=\dim\Gamma(X,\sox)-1=0$
which is impossible because $X$ has dimension at least $1$. 

Thus we may assume $\L\not\isom\sox$ and apply the above lemma. 
There is an $n$
so that $\L^{\tensor n}$ is generated by its global sections. By the
above lemma $H^0(X,(\L^{\tensor n})^{\dual})=0$. Since the collection
of invertible sheaves forms a group and $\dual$ is the inverse operation
it follows trivially that
$(\L^{\tensor n})^{\dual}\isom(\L^{\dual})^{\tensor n}$ and
hence $\Gamma(X,(\L^{\dual})^{\tensor n})=0$.
Suppose $\L^{\dual}$
has a nonzero global section $s$. Let $p\in X$ be a point so that 
$s_p\neq 0$. It follows that $s\tensor\cdots\tensor s\neq 0$ in 
$(\L^{\dual}_p)^{\tensor n}$. Thus $s$ defines a nonzero global section
$s\tensor\cdots\tensor s$ of $(\L^{\dual})^{\tensor n}$. [This last statement
is a bit subtle because the tensor product is the sheaf associated to
a certain presheaf so we don't know, {\em a priori}, that $s\tensor\cdots\tensor s$
maps to something nonzero under the $\theta$ of (II, Defn 1.2). But
if $\theta(s\tensor\cdots\tensor s)=0$ then $0=\theta(s\tensor\cdots\tensor s)_p=(s\tensor\cdots\tensor s)_p$
so $\theta$ is not injective on stalks contradicting the comment after
(II, Defn 1.2).] Thus if $H^0(X,\L^{\dual})\neq 0$ then 
$H^0(X,(\L^{\tensor n})^{\dual}) \neq 0$, a contradiction. It follows
that $H^0(X,\L^{\dual})=0$, as desired. 


\section{Exercise III.7.3}
{\em Let $X=\P^n_k$. Show that $H^q(X,\OX^p)=0$
for $p\neq q$, $k$ for $p=q$, $0\leq p,q\leq n$.}

Our strategy is to use the exact sequence 
$$0\into\OX\into\sox(-1)^{\oplus n+1}\into\sox\into 0$$
of (II, 8.13) along with (II, Ex 5.16 d) to reduce the 
computation of the cohomology of $\OX^p$ to the
computation of the cohomology of $\Lambda^p\sox(-1)^{\oplus n+1}$. 
We then show inductively 
that the cohomology of $\Lambda^p\sox(-1)^{\oplus n+1}$ vanishes for
$p\geq 1$ thus completing the proof. 

We compute the cohomology of $\Omega^r$ inductively on $r$. 

{\em Step 1, $r=0$.} Suppose $r=0$ so $\Omega^r=\sox$. Then
by (III, 5.5) $H^0(X,\sox)=k$ and $H^i(X,\sox)=0$ for $i\geq 1$.
[Part (a) of (III, 5.5) gives $H^0(X,\sox)=k$, part (b) gives
$H^i(X,\sox)=0$ for $0<i<n$ and part (d) gives 
$H^n(X,\sox)\isom H^0(X,\sox(-n-1))^{\dual}=0$.]

{\em Step 2} Show that
$$H^i(\Lambda^r\sox(-1)^{\oplus n+1})=0$$
for $r\geq 1$. 

[Matt Baker pointed out to me that 
$$\Lambda^r\sox(-1)^{\oplus n+1}\isom\sox(-1)^{\oplus \chose{n+1}{r}}.$$
This is reasonable since it is true on stalks. It immediately implies
the vanishing of the cohomology groups. My original more complicated proof
of step 2 is included next anyways.]

{\em Step 2a, $r=1$.} We treat $r=1$ as a special case. 
We must show that $H^i(X,\sox(-1)^{\oplus n+1}=0$ or 
equivalently that $H^i(X,\sox(-1))=0$. This is immediate
from the explicit computations of (III, 5.5). The argument proceeds 
exactly as in step 1. 

{\em Step 2b, $r\geq 2$.} We now assume $r\geq 2$ and proceed inductively
on $n$. Since $r\geq 2$ there is an exact sequence
$$0\into\Lambda^{r-1}\sox(-1)^{\oplus n}\into\Lambda^r\sox(-1)^{\oplus n+1}\into\Lambda^r\sox(-1)^{\oplus n}\into 0.$$
I obtained the map 
$$\Lambda^r\sox(-1)^{\tensor n+1}\into\Lambda^r\sox(-1)^{\tensor n}$$
by carefully applying (II, Ex 5.16d) to the map
$\sox(-1)^{\oplus n+1}\into\sox(-1)^{\oplus n}$. 
But this map just turns out to be locally defined by
$x_n\mapsto 0$ where $x_0,\ldots,x_n$ are local coordinates
for $\sox(-1)$. Then $x_{i_0}\wedge\cdots\wedge x_{i_r}$ 
maps to $0$ if some $i_k=n$ and itself otherwise. 
The map 
$$\Lambda^{r-1}\sox(-1)^{\oplus n}\into\Lambda^r\sox(-1)^{\oplus n+1}$$
identifies $\Lambda^{r-1}\sox(-1)^{\oplus n}$ with the kernel of the
next map. The kernel of the next map is locally generated by all  
``monomials'' which contain an $x_n$. Since $r\geq 2$ we can identify
$\Lambda^{r-1}\sox(-1)^{\oplus n}$ with this kernel by just removing
the $x_n$ off of the wedge product. 
[This is not rigorous enough!]

By induction on $n$ we have that
$$H^i(\Lambda^{r-1}\sox(-1)^{\oplus n})=H^i(\Lambda^r\sox(-1)^{\oplus n})=0$$ 
for all $i$ (we will do the base case $n=1$ in just a moment). 
Thus, by the long exact sequence of cohomology we see that 
$H^i(\Lambda^r\sox(-1)^{\oplus n+1})=0$ for all $i$. 
For $n=1$, since $r\geq 2$
it follows that  $\Lambda^{r-1}\sox(-1)=\sox(-1)$ or $0$
and $\Lambda^r\sox(-1)=0$ and these both have trivial cohomology
as computed above. 

{\em Step 3.} The final step is to obtain the long exact sequence
$$\cdots H^i(\Omega^r)\into H^i(\Lambda^r\sox(-1)^{\oplus n+1})\into H^i(\Omega^{r-1})\into \cdots$$ 
then apply step 2 and the induction hypothesis (we are inducting
on $r$, the base case was established in step 1) to 
calculate $H^i(\Omega^r)$ for all $i$. 

Suppose $r\geq 1$, then by (II, 8.13) we have an exact sequence
$$0\into\OX\into\sox(-1)^{\oplus n+1}\into\sox\into 0.$$
By (II, Ex. 5.16d), $\Lambda^r\sox(-1)^{\oplus n+1}$ has a filtration
$$\Lambda^r\sox(-1)^{\oplus n+1}=F^0\supseteq F^1\supseteq\cdots\supseteq F^r\supseteq F^{r+1}=0$$
with quotients
$$F^p/F^{p+1}\isom\Omega^p\tensor\Lambda^{r-p}\sox=
\begin{cases} 0&\text{if $r-p\geq 2$}\\
              \Omega^p&\text{if $r-p$ is $0$ or $1$}\end{cases}$$
Thus $$\Lambda^r\sox(-1)^{\oplus n+1}=F^0=\cdots=F^{r-1}$$ and the filtration
becomes $$\Lambda^r\sox(-1)^{\oplus n+1}\supset F^r\supset F^{r+1}=0$$ 
with $\Lambda^r\sox(-1)^{\oplus n+1}/F^r\isom \Omega^{r-1}$ and 
$F^r\isom\Omega^r$. 
This gives an exact sequence
$$0\into\Omega^r\into\Lambda^r\sox(-1)^{\oplus n+1}\into\Omega^{r-1}\into 0.$$
The associated long exact sequence of cohomology gives for each $i$ 
an exact sequence 
$$H^i(\Lambda^r\sox(-1)^{\oplus n+1})\into H^i(\Omega^{r-1})
\into H^{i+1}(\Omega^r)\into H^{i+1}(\Lambda^r\sox(-1)^{\oplus n+1}).$$
But by step 2 the groups $H^i(\Lambda^r\sox(-1)^{\oplus n+1})$ all vanish.
Thus $H^i(\Omega^{r-1})\isom H^{i+1}(\Omega^r)$. By induction on 
$r$ this shows that 
$$H^i(\Omega^r)=\begin{cases}0&\text{if $i\neq r$}\\
                             k&\text{if $i=r$}\end{cases}.$$
This completes the proof. 
\end{document}
