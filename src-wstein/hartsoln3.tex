\documentclass[12pt]{article}
\author{William A. Stein}
\title{Algebraic Geometry Homework\\
II.5, 1,2,3,4,5,7,8}

\font\grmn=eufm10 scaled \magstep 1

\renewcommand{\a}{\mbox{\bfseries A}}
\newcommand{\gp}{p}
\newcommand{\ga}{a}
\newcommand{\so}{\mathcal{O}}
\newcommand{\sox}{\mathcal{O}_X}
\newcommand{\soy}{\mathcal{O}_Y}
\newcommand{\se}{\mathcal{E}}
\renewcommand{\sf}{\mathcal{F}}
\newcommand{\sg}{\mathcal{G}}
\newcommand{\shom}{\mathcal{H}\mbox{\rm om}}
\renewcommand{\hom}{\mbox{\rm Hom}}
\newcommand{\tensor}{\otimes}
\newcommand{\tens}{\otimes}
\newcommand{\isom}{\cong}
\renewcommand{\dim}{\mbox{\rm dim}}
\newcommand{\z}{\mbox{\bfseries Z}}
\newcommand{\into}{\rightarrow}
\newcommand{\res}{\mbox{\rm res}}
\newcommand{\ru}{|_U}
\newcommand{\ann}{\mbox{\rm Ann}}
\newcommand{\coker}{\mbox{\rm Coker}}


\newtheorem{prob}{Problem}
\newtheorem{theorem}{Theorem}
\newtheorem{prop}{Proposition}

\newcommand{\ov}{\overline{\varphi}}
\renewcommand{\phi}{\varphi}
\newcommand{\fh}{f^{\#}}
\newcommand{\proj}{Proj \hspace{.01in}}
\newcommand{\spec}{Spec \hspace{.01in}}
\newcommand{\proof}{\mbox{\sc Proof.\hspace{.1in}}}
\date{April 4, 1996}
\begin{document}
\maketitle

\begin{prob}[II.5.1]
Let $(X,\sox)$ be a ringed space, and let $\se$ be a locally
free $\sox$-module of finite rank. We define the dual of
$\se$, denoted by $\check{\se}$, to be the sheaf
$\shom_{\sox}(\se,\sox)$. 

(a) Show that $(\check{\se})\check{}\cong\se$.

(b) For any $\sox$-module $\sf$, $\shom_{\sox}(\se,\sf)
\cong \check{\se} \tensor_{\sox} \sf$. 

(c) For any $\sox$-modules $\sf, \sg,$
$\hom_{\sox}(\se\tensor\sf,\sg)\cong
\hom_{\sox}(\sf,\shom_{\sox}(\se,\sg))$. 

(d) (Projection Formula). If $f:(X,\sox)\rightarrow(Y,\soy)$
is a morphism of ringed spaces, if $\sf$ is an $\sox$-module,
and if $\se$ is a locally free $\soy$-module of finite
rank, then there is a natural isomorphism 
$f_{*}(\sf\tensor_{\sox}f^{*}\se)\cong
f_{*}(\sf)\tensor_{\soy}\se$. 


\end{prob}
\proof
(a) 
A free module of finite rank is {\em canonically}
isomorphic to its double-dual via
$\check{m}(\lambda)=\lambda(m)$
where $m\in M$, $\lambda\in \check{M}$, and
$\check{m}\in (\check{M})\check{}$. 
Let $U$ be an open set on which $\se|U$ is a
free $\sox$-module of finite rank. Define a
map $\phi:\se|_U\into(\check{\se})\check{}|_U$
by, for all $V\subseteq U$, $\se(V)\into
(\check{\se})\check{}(V)$ is the isomorphism described
above. 
Since the isomorphisms are canonical, we can patch on 
intersections and define a global isomorphism. 

(b) As above we may assume $\se$ is a free $\sox$-module.
Let $e_1,\ldots,e_n$ be a basis for $\sf$ and let
$e_1^{*},\ldots,e_n^{*}$ be the corresponding dual
basis. Let $U$ be an open subset of $X$. 
Define $\phi_U:\shom_{\sox}(\se,\sf)|_U\into
(\hom(\se,\sox) \tensor_{\sox}\sf)|_U$ by 
$f\mapsto\sum_{i=1}^{n}e_i^{*}\tens f(e_i).$
Define $\psi_U:
(\hom(\se,\sox) \tensor_{\sox}\sf)|_U \into 
\shom_{\sox}(\se,\sf)|_U$ by
$f\tens a\mapsto(x\mapsto f(x)a)$. 
For convenience of notation write $\phi=\phi_U$ and
$\psi=\psi_U$. 
Let $f\tens a\in (\hom(\se,\sox) \tensor_{\sox}\sf)|_U$.
Then $\phi\circ\psi(f\tens a)
=\phi(x\mapsto f(x)a)=\sum_{i=1}^{n} e_i^{*}\tens
f(e_i)a=\sum_{i=1}^{n}e_i^{*}f(e_i)\tens a
=f\tens a$. Let $f\in\shom_{\sox}(\se,\sf)$, then
$\psi\circ\phi(f)=\psi(\sum_{i=1}^{n}e_i^{*}\tens f(e_i))
=(x\mapsto\sum_{i=1}^{n}e_i^{*}(x)f(e_i))
=(x\mapsto f(\sum_{i=1}^{n}e_i^{*}(x)e_i))
=(x\mapsto f(x))$.
Thus $\phi$ and $\psi$ are inverse bijective 
homomorphisms, hence ring isomorphisms, and since they
respect the restriction maps we see that the 
corresponding sheaves are isomorphic.

(c) On each open set define $\phi|_U:
\hom_{\sox}(\se\tensor\sf,\sg)\into
\hom_{\sox}(\sf,\hom_{\sox}(\se,sg))$ by
$f\mapsto(a\mapsto(e\mapsto f(e\tens a)))$. 
(For notational convenience we omit the sheaf restrictions.)
If $\phi(f)=0$ then the map 
$(a\mapsto(e\mapsto f(e\tens a)))$ is $0$ 
so $f$ is the zero map, hence $\phi$ is injective.
Let $f\in \hom_{\sox}(\sf,\hom_{\sox}(\se,sg))$. Define
$g\in\hom_{\sox}(\se\tensor\sf,\sg)$ by
$g(a\tens b)=(f(b))(a)$.
Then $\phi(g)=(a\mapsto (e\mapsto g(e\tens a)))
=(a\mapsto(e\mapsto(f(a))(e))=(a\mapsto f(a))=f$
so $\phi$ is surjective. Thus $\phi$ is the
desired isomorphism which, since $\phi$ evidently
commutes with the restriction maps, induces an
isomorphism of sheaves.

(d)
First we consider the case when
$\se\isom\soy^n$.
One one hand, 
\begin{eqnarray*}
f_{*}(\sf)\tensor_{\soy}\se
 &\isom& f_{*}(\sf)\tensor_{\soy}\soy^n \\
&\isom& \oplus_{i=1}^{n} (f_{*}(\sf)\tensor_{\soy}\soy)\\
&\isom& \oplus_{i=1}^{n} f_{*}(\sf).\\
\end{eqnarray*}

On the other hand,
\begin{eqnarray*}
\sf\tensor_{\sox}f^{*}\se &=&\sf\tensor_{\sox}f^{*}(\soy^n)\\
&\isom&\sf\tensor_{\sox}(f^{-1}(\soy^n)\tensor_{f^{-1}\soy}\sox)\\
&\isom&\sf\tensor_{\sox}((f^{-1}(\soy)^n\tensor_{f^{-1}\soy}\sox)\\
&\isom&\sf\tensor_{\sox}(\sum_{i=1}^{n}
f^{-1}\soy\tensor_{f^{-1}\soy}\sox)\\
&\isom&\sf\tensor_{\sox}(\sox^n)\\
&\isom&(\sf\tensor_{\sox}\sox)^n=\sf^n
\end{eqnarray*}
where $f^{-1}(\soy^n)\isom f^{-1}(\soy)^n$ since
$f^{-1}$ is a left adjoint functor hence commutes
with direct sums (which are a right universal
construction). 

Putting this together we have that
$$f_{*}(\sf\tensor_{\soy}f^{*}(\se))
=f_{*}(\sf^n)=f_{*}(\oplus_{i=1}^{n}\sf)
=\oplus_{i=1}^{n}f_{*}(\sf).$$  

In general, we construct isomorphisms as above
on an open cover then, since all of the isomorphisms 
are canonical, the isomorphisms match up on
the intersections so we can glue to obtain an 
isomorphism. 

\begin{prob}[II.5.2]
Let $R$ be a discrete valuation ring with quotient field
$K$, and let $X=\spec R$. 

(a) To give an $\sox$-module $\sf$ is equivalent to giving
an $R$-module $M$, a $K$-vector space $L$, and a
homomorphism $\rho:M\tensor_R K\rightarrow L$. 

(b) That $\sox$-module is quasi-coherent if and only
if $\rho$ is an isomorphism. 
\end{prob}
\proof
(a) First suppose we are given an $\sox$-module $\sf$. Since $R$
is a DVR, $X$ has exactly two nonempty open sets,
$X$ and the set consisting of the generic point, 
$\{\xi\}$. Let $M=\Gamma(\sf,X)$ and let 
$L=\Gamma(\sf,\{\xi\})$. 
Since $\Gamma(\sox,X)=R$ and $\Gamma(\sox,\{\xi\})=K$, $M$ is
an $R$-module and $L$ is a $K$-vector space. Let
$g:M\rightarrow L$ be the restriction map. Define
$\rho$ by $\rho(m\tens\alpha)=\alpha\cdot{}g(m)$. 
Since $g$ is a homomorphism, so is $\rho$. 

Now suppose we are given an $R$-module $M$, a $K$-vector
space $L$, and a homomorphism $\rho:M\tensor_R K\rightarrow L$.
Define an $\sox$-module $\sf$ as follows. Let 
$\Gamma(\sf,X)=M$ and $\Gamma(\sf,\{\xi\})=L$. Define
the restriction map $g:M\rightarrow L$ by
$g(m)=\rho(m\tens 1)$. We just need to check that
$g$ is a valid restriction map. Let $r\in R, m\in M$,
then $g$ is a valid restriction map iff
$r\cdot g(m)=g(rm)$, that is, when
$r\cdot \rho(m\tens 1)=\rho(rm\tens 1)=\rho(m\tens r)$. 
So we must verify that the given homomorphism $\rho$ is 
$R$-linear. But there is absolutely no reason why this should
be the case! (For example, let $M=L=K$, then 
$\rho:K\rightarrow K$ and it is easy to construct nontrivial
homomorphisms of the additive group of a field). 
I think the problem is imprecisely stated. It 
should be assumed throughout that $\rho$ is $K$-linear.

(b) First suppose $\sf$ is quasi-coherent. Then 
$\sf=\tilde{M}$ so
Proposition 5.1 implies that
$L=\Gamma(\tilde{M},\{\xi\})=(R-0)^{-1}M\cong
M\tensor_R K$. Thus $\rho$ must be an isomorphism.
Conversely, if $\rho$ is an isomorphism, we see that
$\sf\cong \tilde{M}$ since they are the same on
each open set and the restriction map is the 
same. 

\begin{prob}[II.5.3]
Let $X=\spec A$ be an affine scheme. Show that the functors
$\tilde{}$ and $\Gamma$ are adjoint, in the
following sense: for any $A$-module $M$, and for any sheaf
of $\sox$-modules $\sf$, there is a natural isomorphism
$$\hom_A(M,\Gamma(X,\sf))\cong\hom_{\sox}(\tilde{M},\sf).$$
\end{prob}
\proof

Define a homomorphism $F:\hom_A(M,\Gamma(X,\sf))\into
\hom_{\sox}(\tilde{M},\sf)$ as follows. Send
a ring homomorphism $\phi:M\into\Gamma(X,\sf)$
to the morphism of sheaves $F(\phi):\tilde{M}\into\sf$.
It suffices to define $F(\phi)$ on distinguished
open sets (Eisenbud \& Harris, page 13). 
For $f\in A$ let $F(\phi)_{D(f)}$ be the map
$${m\over f^n}\mapsto {1\over f^n}
\cdot\res_{X,D(f)}(\phi(m))$$
where $\res_{X,D(f)}:\sf(X)\into\sf(D(f))$ 
is the restriction map of $\sf$. 
$F(\phi)_{D(f)}$ is a well-defined homomorphism since
both $\phi$ and $\res_{X,D(f)}$ are homomorphisms and 
$\phi$ is an $A$-module homomorphism. Next note 
that $F(\phi)$ commutes with the
restriction maps since each $\res_{X,D(f)}$ does. 
To see that $F$ is injective suppose $\phi$ and
$\psi$ are two homomorphisms $M\into\Gamma(X,\sf)$.
If $F(\phi)=F(\psi)$ then, in particular,
$\phi=F(\phi)_{X}=F(\psi)_{X}=\psi$.
To see that $F$ is surjective, let $\phi\in\hom_{\sox}
(\tilde{M},\sf)$. Define $\psi:M\into\Gamma(X,\sf)$
by letting $\psi=\phi_X$, that is, by taking the 
induced map on global sections. 
Then, for $f\in A$, 
$F(\psi)_{D(f)}:\tilde{M}(D(f))\into\sf(D(f))$ is
the map
$({m\over f^n}\mapsto {1\over f^n}\res_{X,D(f)}\circ\phi_X(m))$
which, since $\phi$ is a morphism of sheaves, equals
$=({m\over f^n}\mapsto {1\over f^n}\phi_{D(f)}(m)
=\phi_{D(f)}({m\over f^n})=\phi_{D(f)}$.
(We are just using the fact that $\phi$ commutes with
the appropriate restriction maps.)
Thus $F(\psi)$ agrees with $\phi$ on a basis for $X$
hence $F(\psi)=\phi$. This shows that $F$ is surjective.
So $F$ is an isomorphism, as required.

\begin{prob}[II.5.4]
Show that a sheaf of $\sox$-modules $\sf$ on a scheme $X$ 
is quasi-coherent
if and only if every point of $X$ has a neighborhood $U$, such
that $\sf|_{U}$ is isomorphic to a cokernel of a morphism of
free sheaves on $U$. If $X$ is noetherian, then $\sf$ is coherent
iff it is locally a cokernel of a morphism of free sheaves
of finite rank.
\end{prob}
\proof
Suppose first that $\sf$ is quasi-coherent. Let $x\in X$. 
Then there is an affine open neighborhood
$U=\spec A$ of $x$ such that $\sf|_U\isom \tilde{M}$,
$M$ an $A$-module. It suffices to show that $M$ is
isomorphic to a cokernel of a morphism of finitely
generated $A$-algebras. Indeed, if 
$\phi:A^{(I)}\into A^{(J)}$ then 
$\coker(\tilde{\phi})=\tilde{A^{(J)}}/\phi(A^{(I)})\tilde{}
=(A^{(J)}/\phi(A^{(I)}))\tilde{}$
since, for all $f\in A$, 
$(A^{(J)})_f/\phi(A^{(I)})_f
=(A^{(J)}/\phi(A^{(I)}))_f$
so that they agree on a basis. 

Let $A^{|M|}$ be the free $A$-module on the elements
of $M$. 
Let $\phi:A^{|M|}\into M$ be the natural map.
Similiary, let $\psi:A^{|\ker(\phi)|}\into\ker(\phi)
\subseteq A^{|M|}$ be the natural map.
Then $\coker(\psi)\isom A^{|M|}/\ker(\phi)\isom M$,
as required.

Now assume the $\sf$ is coherent. We proceed as above
but now $M$ is a finitely generated $A$-module,
generated by $e_1,\ldots,e_n$, say, and we must
show that $M$ is the cokernel of a morphism of
free modules of finite rank. Let $\phi:A^n\into M$
be the map which takes the $i$th generator
$(0,\ldots,1,\ldots,0)$ of $A^n$ to $e_i\in M$. 
Then $\ker(\phi)$ is a submodule of $M$ so, since
$M$ is neotherian (any finitely generated module
over a noetherian ring is noetherian), $\ker(\phi)$
is finitely generated. Let $f_1,\ldots,f_m$ be
a generating set. Let $\psi:A^m\into\ker(\phi)$ be
the surjection defined by sending the $i$th
basis element of $A^m$ to $f_i$. Then 
$\coker(\psi)\isom A^n/\psi(A^m)\isom A^n/\ker(\phi)\isom M$.
Thus $M$ is isomorphic to a cokernel of a morphism
of free sheaves of finite rank. 
 




\begin{prob}[II.5.5]
Let $f:X\rightarrow Y$ be a morphism of schemes.

(a) Show by example that if $\sf$ is coherent on $X$, then
$f_{*}\sf$ need not be coherent on $Y$, even if $X$ and $Y$
are varieties over a field $k$.

(b) Show that a closed immersion is a finite morphism.

(c) If $f$ is a finite morphism of neotherian schemes, and if 
$\sf$ is coherent on $X$, then $f_{*}\sf$ is coherent on $Y$.
\end{prob}
\proof
(a) Lex $k$ be a field, let $X=\spec (k[x]_x)$,
$Y=\spec (k[x])$ and $\sf=\sox$. Then 
$f_{*}(\sf)(U)=\sf(f^{-1}(U))$, so
$f_{*}(\sf)=(k[x]_x)^{\tilde{}_Y}$.
But $(k[x]_x)^{\tilde{}_Y}$ is not a coherent sheaf
of $\soy$-modules. Indeed, if it were, there would
be a distinguished neighborhood $D(f)$ of $0$ so that
$(k[x]_x)^{\tilde{}_Y}|_{D(f)}$ 
is a finitely generated module over $k[x]$ (here I'm
using the fact that over open set is distinguished
in the topology of $\spec(k[x])$). But, 
$k[x]_f$ is never a finitely generated module over
$k[x]$ for any $f$ of degree at least $1$. For $k[x]_f$
contains elements of arbitrarily small degree whereas
the degrees of elements of $k[x]\{\alpha_1,\ldots,\alpha_n\}$
are bounded below. ($k[x]\{\alpha_1,\ldots,\alpha_n\}$
is the $k[x]$-module generated by $\alpha_1,\ldots,\alpha_n$.)

(b) Let $f:Y\into X$ be a closed immersion. Let $U=\spec(A)
\subseteq X$ and let $W=f^{-1}(U)\subseteq Y$. Then
$W$ is a a scheme (give it the induced scheme structure as
an open subset of $Y$). Furthermore, 
$f(W)=U\cap f(Y)$ is a relatively closed subset of $U$,
that is, a closed subset of $\spec(A)$. The
map $\fh:\so_{U}\into\so_{W}$ is surjective since
$\fh:\sox\into\soy$ is surjective and surjectivety
is a local property. Thus $W$ is a closed subscheme
of $\spec(A)$ so, by Corollary 5.10, $W\cong\spec(A/I)$
for some ideal $I$ of $A$. Since $A/I$ is a finitely
generated $A$ module (generated by $1+I$), $f$
is a finite morphism.

(c) Let $U=\spec(A)\subseteq Y$ be an affine open subset
of $Y$. Then, since $f$ is finite, $f^{-1}(U)=\spec(B)$
is affine with $B$ is a finitely generated $A$-module. 
By Proposition 5.4 it suffices to show that 
$f_{*}(\sf)|_U$ is $\tilde{M}$ for some finitely
generated $A$-module $M$. Now by Proposition 5.4,
since $f^{-1}(U)$ is affine and $B$ is noetherian,
$\sf|_{f^{-1}(U)}=\tilde{M}$ for some finitely
generated $B$ module $M$. 
But 
$f_{*}(\sf)|_U=(f|_{f^{-1}(U)})_{*}\sf|_{f^{-1}(U)}
=(f|_{f^{-1}(U)})_{*}(\tilde{M})=\tilde{(A^M)}$
where the last equality follows from Proposition 5.2(d).
Since $B$ is a finite module over $A$ and $M$
is a finite module over $B$, it follows that $M$
is a finite module over $A$ which completes the proof. 

\begin{prob}[II.5.7]
Let $X$ be a noetherian scheme, and let $sf$ be a coherent sheaf.

(a) If the stalk $\sf_x$ is a free $\sox$-module for some point
$x\in X$, then there is a neighborhood $U$ of $x$ such that $\sf|_U$ 
is free. 

(b) $\sf$ is locally free iff its stalks $\sf_x$ are free $\sox$-modules
for all $x\in X$.

(c) $\sf$ is invertible iff there is a coherent sheaf $\sg$ such
that $\sf\tensor\sg\isom\sox$. 
\end{prob}
\proof
(a) Let $U=\spec(A)$ be a neighborhood of $x$ so that
$\sf\ru=\tilde{M}$, $M$ a finitely generated $A$-module.
Then $\sf_x=M_x$ so we have reduced the problem to
the following purely algebraic result.
\begin{prop}
If $M$ is a finitely generated free $A$ module and there
is a prime $\wp$ of $A$ such that $M_{\wp}$ is a free
$A_{\wp}$-module, then there exists $f\in A$ such that
$f\not\in\wp$ and $M_f$ is a free $A_f$-module. 
\end{prop}

Once we have proven this we will know that $\sf|_{D_U(f)}$
is free since Proposition 5.1 asserts that 
$\tilde{M}(D(f))\cong M_f$.  

\proof (of Proposition) Let $a_1,\ldots,a_n\in M$ be a free
basis of $M_{\wp}$ over $A_{\wp}$ (we can clear denominators
so that we may assume all $a_i$ lie in $M$.) Let
$b_1,\ldots,b_m\in M$ be a generating set for $M$
over $A$. For each $i$ we can write $b_i$ as
an $A_{\wp}$-linear combination of the $a_i$. Clearing
denominators we see that $d_i b_i\in A\{a_1,\ldots,a_n\}$
for some $d_i\not\in\wp$. Let $f=\prod d_i$, then 
$f\not\in\wp$ and $a_1,\ldots,a_n$ have 
$A_f$-span including all of the $b_i$,  and thus
including $M$, and thus including $M_f$. But
$a_1,\ldots,a_n$ is free over $A_{\wp}$ hence over
$A_f$ since $A_f\subseteq A_{\wp}$. [This proposition 
is in Bourbaki, {\em Commutative Algebra},
II.5.1, although the proof is more abstract 
than mine.] 

(b) ($\Longrightarrow$) Let $x\in X$ and let $U=\spec(A)$ be an open
neighborhood of $x$ such that $\sf\ru=\tilde{M}$
with $M$ a free $A$ module. Suppose $\wp$ is the
prime of $A$ corresponding to $x$. Then $\sf_x=M_\wp$
which is a free $A_{\wp}$-module. Indeed,
if $e_1,\ldots,e_n$ is a free $A$-basis for $M$,
then it is also a free $A_{\wp}$-basis for
$M_{\wp}$. For if 
${a_1\over b_1}e_1 + \cdots + {a_n\over b_n}e_n=0, 
b_i\not\in \wp$,
then 
$a_1{b\over b_1}e_1+\cdots+a_n{b\over b_n}e_n=0, b=\prod b_i$
so $b{a_i\over b_i}=0$ for each $i$,
so ${a_i\over b_i}=0$ in the localization $A_{\wp}$ 
since $b\not\in\wp$.  

($\Longleftarrow$) By part (a) every point has a neighborhood
on which $\sf$ is free. Therefore $X$ can be covered by 
open affines on which $\sf$ is free so $\sf$ is locally free. 

(c) ($\Longrightarrow$)
Let $\sg=\check{\sf}=\shom(\sf,\sox)$, we will show that 
$\sf\tensor\sg\isom\sox$. To define a morphism
$\phi:\sf\tensor\shom(\sf,\sox)\rightarrow\sox$  
it is enough to define $\phi$ on the presheaf
$(U\mapsto \sf(U)\tensor_{\sox(U)}\shom(\sf(U),\sox(U))$.
Define $\phi$ by $a\tens f\mapsto f(a)$. Thus $\phi$ commutes
with the restrictions so $\phi$ defines a valid morphism
of sheaves. 

Let $U$ be an open affine subset of $X$ such that $\sf(U)$
is a free $\sox(U)$-module of rank 1 with basis $e_0$. 
Then $\shom(\sf(U),\sox(U))$ has basis $e_0^{*}$ where
$e_0^{*}(e_0)=1$. Thus $\shom(\sf(U),\sox(U))$ is a free
$\sox(U)$-module of rank 1. Thus every element of
$\sf(U)\tensor \hom(\sf(U),\sox(U))$ can be written in the
form $a\tens f$ (as opposed to as a sum of such products). 
Now $\phi_U(a\tens f)=0$ implies $f(a)=0$ which implies
$a=0$ or $f=0$ so $a\tens f=0$, so $\phi_U$ is injective.
Since $\phi_U(ae_0\tens e_0^{*})=a$, $\phi_U$ is surjective.
Thus $\phi_U$ is an isomorphism.

Use the definition of locally free of rank 1 to cover 
$X$ by affine open sets $U$ such that $\sf\ru$ is
a free $\sox\ru$ module of rank 1. Then, by Proposition
5.1 (c) and an argument like that for (b) above, any
distinguished open subset of $U$ has a $\sf$-sections 
free of rank 1. Since $\phi$ is an isomorphism on each
of these distinguished open sets (use the argument in the
above paragraph) and these distinguished open sets form
a basis for the topology on $X$ it follows that $\phi$
must be an isomorphism. 

($\Longleftarrow$) Because of part (b) above it suffices
to show that $\sf_x$ is free of rank one for each $x\in X$. 
Since $\sf_x\tensor_{\so_x}\sg_x
=(\sf\tensor\sg)_x\isom\so_{X,x}$
the problem is reduced to the following purely
algebraic statement.

\begin{prop} Let $M$ and $N$ be finitely generated modules
over a local ring $(A,m)$ and suppose that 
$M\tensor_A N\isom A$. Then $M$ is free of rank 1.  
\end{prop}
\proof
Let $k=A/m$. Let $\nu:M\tensor_A N\into A$ be
the given isomorphism. Then, taking the product
with the identity, we get an isomorphism
$\nu\tensor 1_k:(M\tensor_A N)\tensor_A k\into 
A\tensor_A k\cong k$ (it is obvious that
$\nu\tensor 1_k$ is surjective, but it is 
not at all obvious that it is injective,
for this see the Bourbaki reference below.)
Thus $k\cong M\tensor_A N\tensor_A k
=M\tensor_A(k\tensor_A k)\tensor_A N
=(M\tensor_A k)\tensor_A(N\tensor_A k)
=(M\tensor_A k)\tensor_k(N\tensor_A k)
=(M/mM)\tensor_k (N/mN)$
so, since the $k$-rank of 
$(M/mM)\tensor_k (N/mN)$
is $1$ and is the product of
the ranks of $M/mM$ and $N/mN$, each has
rank 1. In particular, $M/mM$ is monogeneous (generated
by one element) as an $A/m$-module and hence as
an $A$-module so, 
by Nakayama's lemma, $M$ is monogeneous 
as an $A$-module. Since
$\ann_A(M)$ anihilates $M\tensor_A N=A$ as well,
it is $0$ (any element of $\ann_A(M)$ would have 
to annihilate the identity of $A$ and hence be $0$). 
Thus $M$ is a free module of rank one over $A$.  
[See Bourbaki, {\em Commutative Algebra}, II.5.4, for a
more general theorem.] 

\begin{prob}[II.5.8]
Again let $X$ be a noetherian scheme, and $\sf$ a coherent sheaf
on $X$. We will consider the function
$$\phi(x)=\dim_{k(x)}\sf_x\tensor_{\sox}k(x)$$
where $k(x)=\sox/m_x$ is the residue field at the point $x$. Use
Nakayama's lemma to prove the following results.

(a) The function $\phi$ is upper semi-continuous, i.e., for any 
$n\in \z$, the set $\{x\in X:\phi(x)\geq n\}$ is closed.

(b) If $\sf$ is locally free, and $X$ is connected, then
$\phi$ is a constant function. 

(c) Conversely, if $X$ is reduced, and $\phi$ is constant, 
then $\sf$ is locally free.
\end{prob} 
\proof
 
(a) We must show that $\{x:\phi(x)\geq n\}$ is closed.
A good way to do this is by showing that 
$\{x:\phi(x)<n\}$ is open. To do this we show that
if $\phi(x)=m$ then there is an open neighborhood $U$ of
$x$ so that, for all $y\in U$, $\phi(y)\leq m$. 
Since we need only look locally, we can assume that
$X=\spec A$, $\sf=\tilde{M}$, $M$ a finitely generated
$A$-module. Note that $\sf_x\tensor_{\sox}k(x)=
M_{\wp}\tensor_{A_{\wp}}A_{\wp}/{\wp A_{\wp}}=
M_{\wp}/{\wp M_\wp}$.   
Let $s_1,\ldots,s_m\in M$ be elements whose images
form a basis for the vector space $M_{\wp}/\wp M_{\wp}$
over $A_{\wp}/\wp A_{\wp}$ (to do this choose a basis
for $M_{\wp}/\wp M_{\wp}$ then clear fractions). Note
that the images of the $s_i$ in fact generate
$M_{\wp}/{\wp M_{\wp}}$ as an $A_{\wp}$-module. 
By Nakayama's lemma the $s_i$ generate 
$M_{\wp}$ as an $A_{\wp}$-module. 
Let $m_1,\ldots,m_k$ be a generating set for
$M$ over $A$. Write $m_j=\sum {a_i\over b_i}s_i$,
$b_i\not\in\wp$, then, if $c_j=\prod b_i$, 
$c_jm_j$ is in the $A$-span of the $s_i$. Let
$f=\prod c_j$. Then $\wp\in D(f)$ and if 
$q\in D(f)$, then $m_1,\ldots,m_k$ all lie
in the $A_q$-span of $s_1,\ldots,s_m$ (since
$c_jm_j$ is in the $A$-span of the $s_i$
and $c_j$ is inverted in $A_q$. Thus $M$ is spanned
by the $s_i$ over $A_q$, so $M_q$ is spanned by
the $s_i$ over $A_q$. It follows that 
$\phi(q)=\dim M_q/qM_q\leq m$ since the images
of the $s_i$ generate $M_q/qM_q$ as a vector
space over $A_q/qM_q=k(q)$. Taking $D(f)$ as
our open neighborhood completes the proof.

(b) Choose $n$ so that some section of $\sf$ 
has rank $n$. Let $U$ be the union of all open sets $W$
such that $\sf|_W\isom\sox^n|_W$. Then $U$ is nonempty.
Let $V$ be the
union of all open sets $W$ such that
$\sf|_W\isom\sox^m|_W$, $m\not=n$. Since $\sf$ is
locally free, $U\cup V=X$. Suppose $x\in U\cap V$, 
then $\sf_x$ has rank $n$ and rank $m\not=n$ (since
rank is preserved under localization), a 
contradiction. Thus $U\cap V=\emptyset$. Since $U$
is nonempty and open, $X-U=V$ is open and $X$ is 
connected, thus we conclude that $V=X-U=\emptyset$. Thus
every point is contained in an open set $W$ such
that $\sf|_W\isom\sox^n|_W$. 

Let $x\in X$ and let $U=\spec(A)$ be an affine open set
containing $x$ such that $\sf|_U\isom\tilde{M}$. By the above
argument, $M$ is a free $A$-module of rank $n$. 
Thus $\phi(x)=\dim_k M_x\tensor_{A_x}A_x/m_x
=\dim_k A_x^n\tensor_k A_x/m_x=\dim_k (A_x/m_x)^n
=\dim_k k^n=n$, as desired.   

(c) Let $x\in X$. By exercise 5.7b it suffices to
show that the stalk $\sf_x$ is free. Since $\sf$
is coherent we can find an affine open set 
$U=\spec(A)$ such that $\sf\ru=\tilde{M}$ for
some finitely generated $A$-module, $x\in U$,
and $A_f$ is reduced for each $f\in A$. Let
$\wp$ be the prime of $A$ corresponding to
$x$. We must show that $M_{\wp}$ is free over
$A_{\wp}$. Let $s_1,\ldots,s_n\in M$ be preimages
of a basis of $M_{\wp}/\wp M_{\wp}$ over
$k(x)=A_{\wp}/\wp A_{\wp}$ (find these
as in part (a)). Then, by Nakayama's lemma, the
$s_i$ generate $M_{\wp}$ over $A_{\wp}$. 

We must show that the $s_i$ are linearly
independent over $A_{\wp}$. It will then follow
that $M_{\wp}$ is free of rank $n$ over $A_{\wp}$.
So suppose 
$${a_1\over b_1}s_1+\cdots+{a_n\over b_n}s_n=0$$
in $M_{\wp}$ with ${a_i\over b_i}\in A_{\wp}$.
Then for each $i$, $b_i\not\in\wp$ and  
$a_i\in A$. Since the $s_i$ are linearly independent
over $A_{\wp}/\wp A_{\wp}$, for each $i$ there
exists $c_i\not\in\wp$ such that ${c_i a_i\over b_i}\in\wp$.
Thus $c_i a_i\in\wp$ so $a_i\in\wp$. 
Let $r$ be as in part (a) so that $q\in D(r)$ implies
the $s_i$ generate at least $M$ over $A_q$. 
By definition there exists $c\in A$ such that
$$c(b_2\cdots b_n a_1 s_1+\cdots+b_1\cdots b_{n-1} a_n s_n)=0$$
in $A$. 
Let $f=rc\prod b_i$, then if $q\in D(f)$, then 
$s_1,\ldots,s_n$ generate $M_q/qM_q$ over $A_q$ and, 
since $M_q/qM_q$ has dimension $n$ 
(since $\phi$ is constant), 
the $s_i$ are actually a basis for $M_q/qM_q$ over
$A_q/qA_q$.

Since $c|f$, $c\not\in q$ so, as above, 
${a_1\over b_1}s_1+\cdots+{a_n\over b_n}s_n=0$ in
$M_q$, so, as above, $a_i\in q$ for each $i$.  
Thus, for all $q\in D(f)$, $a_i\in q$, so
$a_i$ lies in the nilradical of $A_f$ which, since
$A_f$ is reduced, means that $a_i=0$ in $A_f$. 
So $a_i$ maps to $0$ under the map $A_f\into A_{\wp}$. 
Thus $s_1,\ldots,s_n$ are linearly independent
over $A_{\wp}$ so $M_{\wp}$ is free of rank $n$
over $A_{\wp}$. Applying exercise (5.7b) then completes
the proof.
 
    


\end{document}
