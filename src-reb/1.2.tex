\def\Z{{\bf Z}}
{\bf Hartshorne, Chapter 1}
Answers to exercises. \hfill REB 1994
\item{2.1} $a$ is homogeneous and so defines a cone in 
$A^{n+1}$. $f$ vanishes on all the elements of this cone
(including 0 as $f$ has positive degree) so $f^q\in a$
for some $q>0$ by the usual Nullstellensatz. 
\item{2.2} (iii) implies (i) is trivial as $x_i^d\in S_d$. 
Proof that (i) implies (ii):  If $Z(a)$ is empty, then in 
$A^{n+1} $, $Z(a)$ must be empty or $(0,\ldots,0)$, so $\sqrt a$
must be $S$ or $\oplus_{d>0}S_d$. 
Proof that (ii) implies (iii):
 $\sqrt a $ contains $x_i$, so there is some $m$ with
$x_i^m\in a$ for all $i$, so $a$ contains $S_{m(n+1)}$ as any monomial of degree
$m(n+1)$ must have $x_i^m$ as a factor for some $i$. 
\item{2.3} (a),(b),(c),(e) are trivial. For (d), clearly $I(Z(a))$
contains $\sqrt a$. As $Z(a)$ is nonempty, any nonzero homogeneous polynomial
vanishing on it must have positive degree. By 2.1, this implies that
$f^q\in a$.
Therefore $I(Z(a))$ is contained in $\sqrt a$ as it is a
homogeneous ideal.  
\item{2.4a} Follows from 2.3d,e, and 2.2.
\item{2.4b} If $Y=Y_1\cup Y_2$, then $I(Y)=I(Y_1)\cap I(Y_2)\supset
I(Y_1)I(Y_2)$. Therefore if $I(Y)$ is prime, $I(Y)$ must be either
$I(Y_1)$ or $I(Y_2)$, so $Y$ is $Y_1$ or $Y_2$. 
On the other hand if $Y$ is not prime, then $ab\in I(Y)$,
with $a\notin I(Y)$, $b\notin I(Y)$. Therefore $Y$ is the union of the proper
subsets $Y\cap Z(a)$, $Y\cap Z(b)$
and is therefore not irreducible. 
\item{2.4c} $I(P^n)=0$ which is a prime ideal.
\item{2.5a} $P^n$ can be covered by $n+1$ copies of $A^n$ which is
Noetherian. 
\item{2.5b} See proposition 1.5 and part (a) of this question. 
\item{2.6} S(Y) is the coordinate ring of the cone in $A^{n+1}$
corresponding to $Y$ (assuming $Y$ is nonempty).
$S(Y)_{x_i}$ is the coordinate ring of the cone $Y-(x_i=0)$
if $x_i$ is not identically 0 on $Y$, i.e., $Y_i$ is nonempty. Therefore
the homogeneous part of degree 0 of $S(Y)_{x_i}$ is the coordinate ring of
the cone with $x_i=0$, which is isomorphic to $Y_i$, and therefore
$S(Y)_{x_i}=A(Y_i)[x_i,1/x_i]$ as every element of 
$S(Y)_{x_i}$ is the sum of monomials of the form $(x_i^{\pm n}\times
\hbox{ element of degree 0})$. Therefore 
$Tr.deg.(S(Y)_{x_i}) = Tr.deg. (A(Y_i)+1) = Tr.deg. S(Y)$. 
Therefore $\dim(S(Y))=1+\dim(Y_i)$ whenever $Y_i\ne 0$. 
The $Y_i$'s cover $Y$, so $\dim(Y)=\sup(\dim(Y_i))$, so 
$\dim(S(Y)=1+\dim(Y)$. 
\item {2.7a} $P^n$ is covered by $n+1$ open copies of $A^n$,
so $\dim(P^n)=\sup(\dim(A^n))=n$. 
\item{2.7b} $Y$ is contained in $P^n$, and therefore covered by $n+1$ copies
of $A^n$. In each copy $A_i$ of $A^n$, $\overline{Y\cup A_i}=
\overline{Y}\cup A_i$ as $A_i$ is open. Hence $\dim(Y\cap A_i)= \dim(
\overline{Y\cap A_i}) = \dim(\overline{Y}\cap A_i)$, and therefore
$\dim(Y)=\sup(\dim(Y\cap A_i)= \sup\dim(\overline{Y}\cap A_i)=
\dim(\overline{Y})$. 
\item {2.8} If $f$ is any homogeneous polynomial of positive degree then the zero set
of $f$ has dimension $n-1$ as it has this dimension on some affine subsets
and is a proper closed subset of $P^n$. Also $f$ is irreducible,
so the homogeneous ideal generated by it is prime (as rings of polynomials
are U.F.D.'s so irreducibles are primes) so its variety is irreducible. 
Conversely if $Y$ is any proper closed subset of
$P^n$ then there is some homogeneous polynomial $f$ vanishing on $Y$
which we can assume to be irreducible because $Y$ is irreducible
(so some factor of $f$ must also vanish on $Y$ if $f$ is not irreducible).
Then the zero set of $f$ is an irreducible $n-1$ dimensional 
closed subset of $P^n$ containing the $n-1$ dimensional closed
subset $Y$, and so must be equal to $Y$ (because any proper closed
subset of an irreducible topological space has smaller dimension). 
\item {2.9a} $\beta g(x_0,\ldots,x_n) = x_0^dg(x_1/x_0,\ldots,x_n/x_0)$
if $g$ is of degree $d$. If $g$ vanishes on $Y$ then $\beta g$ vanishes
on $\bar Y$, so $I(\bar Y)\supseteq \beta(I(Y))$. If $h$ vanishes on $\bar Y$
then we can assume $h$ is homogeneous. If $g(x_1,\ldots,x_n)=h(1,x_1,\ldots,x_n)$,
then $h=\beta g$, so $I(\bar Y)$ is generated by $\beta(I(Y))$
\item {2.9b} $\{(t,t^2,t^3)\}=Y$, and $I(Y)=(x_2-x_1^2, x_3-x_1^3)$. 
$\beta(x_2-x_1^2)=x_0x_2-x_1^2$ and $\beta(x_3-x_1^3)= x_0^2x_3-x_1^3$. 
But $I(\bar Y)$ contains $x_1x_3-x_2^2$ which is not contained in 
$(\beta(x_2-x_1^2),\beta(x_3-x_1^3))$.
\item{2.10a} Obvious. 
\item{2.10b} They have the same ideal, which is prime if and only 
if they are irreducible.
\item{2.10c} See 2.6. 
\item{2.11a} $I(Y)$ is generated by linear polynomials $\{p_i\}$
if and only if $Y$ is the intersections of the hyperplanes $\{p_i=0\}$. 
\item{2.11b} Any hyperplane in $P^n$ is a copy of $P^{n-1}$, and the intersection of any other hyperplane of $P^n$ with this $P^{n-1}$ is a hyperplane of
the $P^{n-1}$. Therefore any $r$-dimensional linear variety in
$P^n$ is the intersection of $n-r$ hyperplanes and not the intersection
of $n-r-1$ hyperplanes. Therefore its ideal is minimally generated
by $n-r$ linear polynomials. 
\item {2.11c} $Y$ is the intersection of $n-r$ hyperplanes and
$Z$ is the intersection of $n-s$ hyperplanes, so $Y\cap Z$ 
is the intersection of $2n-r-s$ hyperplanes, 
which has dimension at least $n-(2n-r-s)=r+s-n$. In particular it 
is nonempty if $r+s\ge n$. 
\item{2.12a} $\theta$ maps $k[y_0,\ldots,y_N]$ to an integral domain,
so its kernel is a prime ideal. If $f\in k[y_0,\ldots,y_N]$,
$f=f_0+f_i+\cdots$ with $f_i $ of degree $i$, then $\theta(f_i)$ has
degree $di$, so $\theta(f)=0$ if and only if $\theta(f_i)=0$ for all
$i$, and therefore the kernel is also a homogeneous ideal.
\item{2.12b} If $f\in Ker(\theta)$ then $f(M_0,\ldots, M_n)=0$. 
Hence $f$ vanishes on any point $(M_0(a),\ldots ,M_n(a))$, so
$Im(\rho_d)\subseteq Z(a)$. This proves the easy half.  Any monomial
raised to the power of $d$ is a product of monomials of the form
$x_i^d$. Choose any point $(m_0,\ldots,m_N)\in Z(a)$. Some $m_i$ is
nonzero and $m_i^d=\prod_Nm_{j_n}$ where each $m_{j_N}$ corresponds to
some monomial $x_i^d$, hence some $m_i$ corresponding to a monomial
$x_i^d$ is nonzero; say $i=0$.  If $m_{i_1},\ldots, m_{i_n}$
correspond to $x_0^{d-1}x_1, \ldots, x_0^{d-1}x_n$ then put $x_0=1$,
$x_k=m_{i_k}/m_0$, and try to use this to define a map to $P^n$ on the
set with $m_0\ne 0$. We have to show that
$m_0M_i(1,x_1,\ldots,x_n)=m_i$ (where $m_0$ corresponds to $x_0^n$),
i.e., that $m_0M_i(1,m_{i_1}/m_0,\ldots,m_{i_n}/m_0)=m_i$. But this is
true because $x_0^dM_i(1,x_1/x_0,\ldots,x_n/x_0)=M_i(x_0,\ldots,
x_n)$, and therefore $(m_0,\ldots, m_N)$ is the image of
$(x_0,\ldots,x_n)$. Hence $Im(\rho_d)\supseteq Z(a)$.
\item {2.12c} $\rho_d$ is continuous and bijective from $P^n$ to $Z(a)$. 
To show that it is a homeomorphism it is sufficient to show that 
its inverse is continuous on any open set of $Z(a) $ of the
form $m_i\ne 0$ (notation as above) because these open sets cover
$Z(a)$. But this follows from the construction of this inverse 
above. 
\item{2.12d} The 3-tuple embedding of $P^1$ into $P^3$ 
maps $(x_0:x_1)$ to $(x_0^3:x_0^2x_1:x_0x_1^2:x_1^3)$
which is the projective closure of $\{(x_1,x_1^2,x_1^3)\}$ in $P^3$, 
i.e., the twisted cubic curve. 
\item{2.13} The map is given by $(x_0:x_1:x_2)\rightarrow
(x_0^2:x_1^2:x_2^2:x_0x_1:x_1x_2:x_2x_0)$. Any curve in $P^2$ 
is defined by some polynomial $f(x_0,x_1,x_2)=0$, $f$ homogeneous,
and therefore also by the polynomial $f(x_0,x_1,x_2)^2=
g(x_0^2,x_1^2,x_2^2,x_0x_1,x_1x_2,x_2x_0)$ for some polynomial $g$. 
Then some factor of this polynomial $g$ defines a suitable hypersurface containing
the image of the curve $Z$. (This assumes that $P^2$ is isomorphic to its
image which is easy to check (see 2.14 below) once one has defined isomorphisms
of varieties, so that curves in the image of $P^2$ correspond to curves
in $P^2$.)
\item{2.14} The image of $\psi$ is the set $Y$ defined by the equations 
of the form $x_{ab}x_{cd}= x_{ac}x_{bd}$. Proof: the image is clearly 
contained in $Y$. Conversely
if $(x_{00}:x_{10}:\cdots :x_{rs})\in Y$ the we may assume that 
$x_{00}$ (say) is nonzero. But then the point is the image of
$(x_{00}:x_{10}:\cdots :x_{r0})\times(x_{00}:x_{01}:\cdots :x_{0s})\in
P^r\times P^s$. 
\item{2.15a} $(a_0:a_1)\times(b_0:b_1)= (a_0b_0:a_0b_1:a_1b_0:a_1b_1)
=(w:x:y:z)$, and the image of $P^1\times P^1$ is then the 
subvariety $xt-zw=0$ as in 2.14. 
\item{2.15b} $Q$ is isomorphic to $P^1\times P^1$, so we can take
the two families of lines to correspond to point$\times$line
and line$\times$point. (It is easy to check that these
are lines in $Q\subset P^3$; for example the image
of $(a_0:a_1)\times P^1$ is the set of points $(w:x:y:z)\in P^3$
with $a_1w=a_0y$, $a_1x=a_0z$.)
\item{2.15c} The closed subset $x=y$ of $Q$ is not 
one of these lines.
\item{2.16a} $x^2=yw$, $xy=zw$, so $y^2w=xzw$, so $w=0$ or $y^2=xz$.
Hence $Q_1\cap Q_2$ is the intersection of the line
$w=x=0$ and the twisted cubic $x^2=yw, xy=zw, y^2=xz$. 
\item{2.16b} $L\cap C$ is the point $P=(0:0:1)$, so $I(P)=(x,y)$,
but $I(L)+I(C)=(x^2,y)\neq (x,y)$. 
\item{2.17a} By problem 1.8, the intersection of $q$ hypersurfaces
has dimension at least $n-q$. If $a$ can be generated by $q$ elements
then $Z(y)$ is the intersection of $q$ hypersurfaces and therefore has
dimension at least $n-q$ (using problem 2.8).  
\item{2.17b} If $I(Y)$
can be generated by $r$ elements then $Y$ is the intersection 
of their hypersurfaces. 
\item{2.17c} $Y$ is the intersection of $H_1=Z(x^2-wy)$
and $H_2=Z(y^3+wz^2-2xyz)$ as $(xy-wz)^2=w(y^3+wz^2-2xyz)+y^2(x^2-wy)$
and $(y^2-xz)^2=y(y^3+wz^2-2xyz)+z^2(x^2-wy)$, and
$y^3=wz^2-2xyz=y(y^2-xz)+z(wz-xy)$. On the other hand 
$I(Y)$ has no homogeneous elements of degree 0 or 1 and the space
of homogeneous elements of degree 2 is 3 dimensional, so 
any set of generators must have at least 3 elements. 
\item{2.17d} Still an unsolved problem (as far as I know).


\end
